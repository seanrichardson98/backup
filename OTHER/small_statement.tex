\documentclass[12pt]{amsart}                                                 
\usepackage[margin=1in]{geometry}

\begin{document}
\title{Statement of Interest}
\author{Sean Richardson}
\maketitle

First, I will describe some relevant research experience and future career
interests, then I will address my interest in the SMALL program
specifically.\\

My research experience began in Summer of 2017 through the Lewis \& Clark
computer science department with a project on neural networks. Our team ---
research mentor Peter Drake, a fellow research student, and I --- worked
towards an automated process to identify clouds in images of the sky, which
could assist climate scientists. We implemented and trained a convolution
neural network that takes a picture of the sky as input and distinguishes
between clear, thin cloud, and thick cloud pixels with 94\% accuracy. This
was my first introduction to research and I enjoyed it, but over the next
year my interests became less focused on computer science and more focused
on pure mathematics.\\

So, in the Summer of 2018 I moved towards a pure mathematics research
project through Lewis \& Clark. In this project, my research mentor Liz
Stanhope and I considered the abstract geometrical constructions of
‘orbifolds’ (generalizations of a manifold) and asked: if it is only known
at what frequencies an unknown orbifold vibrates at (formally the Laplace
spectra), what properties can we deduce about the orbifold? We found such a
property, which we call “local orientability”. We are currently in the
process of writing a manuscript on the proof of this result. Before this
project, I had never explored one specific area of mathematics so deeply,
but I enjoyed exploring the patterns that arise from the question and to
contribute a tiny additional piece of understanding to mathematics.\\

As a side note, while I have not taken a course in abstract algebra,
orbifolds are defined by having local structure of the topological quotient
of $\mathbb{R}^n$ with respect to some group of isometries. So, in working
with an classifying orbifolds, I had to learn some basic abstract
algebra.\\

The orbifold research project has influenced my future career interests. In
particular, I am now highly interested in mathematics research and teaching
which I now discuss.\\

    Firstly, the orbifold research project has motivated a potential future
    in research. I would like to continue learning mathematics, so I plan
    to pursue mathematics to the Ph.D level and continue research. In
    particular, I have an interest in differential geometry. I am currently
    doing an informal independent study with professor Iva Stavrov on
    Riemannian geometry and relativity. I find both this study of
    Riemannian geometry and the research project on orbifolds to have an
    appealing connection between mathematic formality and visual intuition,
    which in part contributes to my interest in differential geometry.\\

    Additionally, after participating in this study on orbifolds, I now
    would like to pursue a career in teaching. Unexpectedly, I particularly
    enjoyed presenting my research project through various slideshow or
    poster presentations. Specifically, I gave a slideshow presentation to
    the other research students on campus --- primarily biology and
    chemistry students --- in which I described the project. I attempted to
    give an intuitive interpretation of the research question and our
    approach to answer the question while maintaining accuracy. And, I
    believe that the audience left with an intuitive understanding of my
    project and hopefully an appreciation for the natural patterns that are
    captured behind the formality of math. This sharing of intuition is
    something that I try to do in my work as a math, physics, and computer
    science tutor at the Lewis \& Clark tutoring center, and it is something
    that I would like to continue in the future with teaching.\\

Now, to address the SMALL program, the Dynamics and Number Theory project
particularly appeals to me. If I understand the project correctly, the
question itself is fairly simple: can we break down numbers into a form in
which periodicity is equivalent to being in some specific class of
algebraic numbers. But, as stated in the description, it appears answering
this question in part makes use of many different aspects of mathematics; I
noticed that many of these aspects involve topology and manifold theory. As
discussed above, I am interested in differential geometry and I am
intrigued on this apparent connection between a more classical number
theory question and the theories of topology and geometry or other areas of
mathematics.\\

Additionally, I would find having a research experience outside of Lewis \&
Clark College valuable. The three mathematics professors I know best at
Lewis \& Clark and those that would have research opportunities in pure
mathematics all specialize in differential geometry. While I enjoy
differential geometry, I would like to utilize my time as an undergraduate
exploring more parts of mathematics --- perhaps I will find a subject even
more interesting to me that I will pursue. In addition to the mathematics
itself, I would value the experience conducting research in a different
environment which will broaden my understanding of the research process.\\

Finally, I would like to experience living on the east coast --- someplace I have never been.  It appears that Williamstown is close to various forests, trails, and nature in general. As a Pacific Northwesterner, this something I appreciate. I run nearly every day, and it would be nice to have new trails to run, new places to see, and new mathematical puzzles to explore.

\end{document}
