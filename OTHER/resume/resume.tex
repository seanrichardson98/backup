%TODO: Shorten Research Experience section. Change title? Make links clearer? condense academic achievements?

\documentclass{resume} 
\usepackage[left=0.75in,top=0.6in,right=0.75in,bottom=0.6in]{geometry}
\usepackage{multicol}
\usepackage{enumitem}
\usepackage{hyperref}
\newcommand{\tab}[1]{\hspace{.2667\textwidth}\rlap{#1}}
\newcommand{\itab}[1]{\hspace{0em}\rlap{#1}}

\hypersetup{
    colorlinks=true,
    linkcolor=black,
    filecolor=magenta,      
    urlcolor=black,
}

\name{Sean Richardson}
\address{8622 SE 11th Avenue Portland, OR 97202}
\address{seanhrichardson@gmail.com \\ 503-707-9353}

\begin{document}
\begin{rSection}{Education}
    \textbf{Lewis \& Clark College, Portland Oregon} \hfill \textit{September 2016 - May 2020}\\
    Cumulative GPA: 3.973,\ \ \ Math \& CS GPA: 4.0,\ \ \  Physics GPA:
    4.0\\
    Bachelor of Arts in Computer Science and Mathematics\\
    Bachelor of Arts in Physics

    \textbf{Budapest Semesters in Mathematics, Budapest Hungary} \hfill \textit{Summer 2019}\\
    A mathematics study abroad program for advanced undergraduates. Achieved grade A in all courses.
\end{rSection}

\begin{rSection}{Research Experience}
    \textbf{3-Orbifolds and their Laplace Spectra} \hfill
    \textit{Lewis} \& \textit{Clark College, Summer 2018}\\
    A mathematics research project under faculty advisor Liz Stanhope. The result: a proof that two abstract geometric constructions must have the same local orientability to have the same resonance frequencies, resulting in the published paper 
\href{https://www.sciencedirect.com/science/article/pii/S092622451930097X?utm_campaign=STMJ_75273_AUTH_SERV_PPUB&utm_medium=email&utm_dgroup=Email1Publishing&utm_acid=-800555120&SIS_ID=-1&dgcid=STMJ_75273_AUTH_SERV_PPUB&CMX_ID=&utm_in=DM597592&utm_source=AC_30}
{``You can hear the local orientability of an orbifold''}.


\textbf{Identifying Clouds with Neural Networks} \hfill
\textit{Lewis} \& \textit{Clark College, Summer 2017}\\
An interdisciplinary research project in which myself and a peer implemented and trained a convolution neural network that takes a picture of the sky as input and distinguishes between clear, thin cloud, and thick cloud pixels with 94\% accuracy. Worked under faculty advisors Peter Drake and Jessica Kleiss.


\end{rSection}

\begin{rSection}{Academic Achievements} 
    \textbf{Pi Mu Epsilon Honors Society} \hfill \textit{November 2019}\\
 A national mathematics honors society; membership determined by college math faculty.

    \href{https://www.sciencedirect.com/science/article/pii/S092622451930097X?utm_campaign=STMJ_75273_AUTH_SERV_PPUB&utm_medium=email&utm_dgroup=Email1Publishing&utm_acid=-800555120&SIS_ID=-1&dgcid=STMJ_75273_AUTH_SERV_PPUB&CMX_ID=&utm_in=DM597592&utm_source=AC_30}
	{\textbf{Published Academic Paper} \hfill \textit{October 2019} \\
``You can hear the local orientability of an orbifold'' in \textit{Differential Geometry and its Applications}.}

    \textbf{Scholar Athlete of the Year Award} \hfill \textit{May 2019}\\ 
An award bestowed by Lewis \& Clark athletic administration for academic and athletic achievements.

    \textbf{Feynmann Book Award to Outstanding Introductory Physics Student}
	\hfill \textit{May 2018}\\
    Bestowed by college physics faculty to the top sophomore physics student based on academics.

    \textbf{CCSC Conference Presentation} \hfill \textit{October 2017}\\
	Poster presentation of cloud identification research in 2017 CCSC conference; our poster won first prize.

    \textbf{Pamplin Honors Society} \hfill \textit{September 2017} \\
 Membership extended by the Society to seven Lewis \& Clark students annually for academics/leadership.

\end{rSection}
\begin{rSection}{Leadership}
    \textbf{Peer Tutor:} \textit{(Fall 2017-Present)}
Tutor at Lewis \& Clark quantitative resource center for math, physics, and computer science. Requires checking in with all students and looking after center materials.

	\textbf{Guiding Classes:} Trusted to help teach the occasional Lewis \& Clark class in the case of an unavailable professor. Specifically, two quantitative reasoning classes and a Putnam test preparation class.

    \textbf{Grader:} \textit{(Fall 2019, Spring 2020)} A grader for various math and physics classes at Lewis \& Clark. 

    \textbf{Running Athlete:} \textit{(Fall 2016-Spring 2020)} Four year varsity athlete for Lewis \& Clark cross country and track teams. Took on leadership role junior and senior years as most experienced distance runner.


\end{rSection}

\end{document}

%Programming languages / technical skills section?

% \begin{rSection}{References}
%     \begin{center}
%     \begin{tabular}{l l l}
%         Name         & \hspace{3cm} & Email \\
%     \hline
%     Iva Stavrov  & \hspace{3cm} & istavrov@lclark.edu \\
%     Liz Stanhope & \hspace{3cm} & stanhope@lclark.edu \\
%     Paul Allen   & \hspace{3cm} & ptallen@lclark.edu
%     \end{tabular}
%     \end{center}
% \end{rSection}

%\begin{rSection}{Relevant Courses}
%    \vspace{-1mm}
%    \begin{multicols}{3}
%        \begin{itemize}[topsep=0pt,itemsep=0pt,parsep=0pt,before=\vspace{1cm},after=\vspace{1mm}]
%            \item Algorithm Design: A
%            \item Differential Eqns: A
%            \item Linear Algebra: A
%            \item Multivariable Calculus: A
%            \item Computer Graphics: A
%            \item Advanced Graphics: A
%            \item Number Theory: A
%            \item Theory of Computation: A
%            \item Real Analysis: A
%        \end{itemize}
%        \vspace{-3mm}
%    \end{multicols}
%    \vspace{-3mm}
%\end{rSection}


% \begin{rSection}{Career Interests}
%     \textbf{Teaching:}
%     I find particular enjoyment in working as a tutor, which motivates my
%     career interest of teaching. I would enjoy teaching at any level, but
%     teaching at the university level best aligns with my research interests.\\
%     \textbf{Further Education \& Research:}
%     I am interested in pursuing education in mathematics to the Ph.D level.
%     I have a particular interest in the mathematical branch of
%     differential geometry, which I would enjoy pursuing research in.
% \end{rSection}
%
%
%\begin{rSection}{Research Experience}
%\textbf{3-Orbifolds and their Laplace Spectra} \hfill \textit{Summer
%2018}\\
%This research project considers abstract geometrical constructions called
%'orbifolds' (a generalization of a manifold) and asks: if it is only known
%at what frequencies some unknown orbifold vibrates at (formally the Laplace
%Spectra), what properties can we deduce about the orbifold? We found such a
%property, which we call ``local orientability''.
%
%\textbf{Identifying Clouds with Convolutional Neural Networks} \hfill
%\textit{Summer 2017}\\
%Our team worked towards an automated process to identify clouds in images
%of the sky, which could assist climate scientists. We
%implemented and trained a convolution neural network that can take a
%picture of the sky as input and distinguish between clear, thin cloud, and
%thick cloud pixels with 94\% accuracy.
%
