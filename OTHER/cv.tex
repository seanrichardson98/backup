\documentclass{resume} 
\usepackage[left=0.75in,top=0.6in,right=0.75in,bottom=0.6in]{geometry}
\usepackage{multicol}
\usepackage{enumitem}
\newcommand{\tab}[1]{\hspace{.2667\textwidth}\rlap{#1}}
\newcommand{\itab}[1]{\hspace{0em}\rlap{#1}}
\name{Sean Richardson}
\address{seanrichardson98@gmail.com}

\begin{document}
\begin{rSection}{Education}
    \textbf{Lewis \& Clark College, Portland Oregon} \hfill \textit{September 2016 -
    Present}\\
    Cumulative GPA: 3.963,\ \ \ Math \& CS GPA: 4.0,\ \ \  Physics GPA:
    4.0\\
    Will graduate in \textit{May 2020} with:\\
    Bachelor of Arts in Computer Science and Mathematics\\
    Bachelor of Arts in Physics\\
    \textbf{Budapest Semesters in Mathematics, Budapest Hungary} \hfill \textit{Summer 2019}\\
    An mathematics study abroad program for advanced undergraduates. In this program, I studied Number Theory (grade: A) and Conjecture \& Proof (grade: A).
\end{rSection}
\begin{rSection}{Courses}
    \vspace{-1mm}
    \begin{multicols}{3}
        \begin{itemize}[topsep=0pt,itemsep=0pt,parsep=0pt,before=\vspace{1cm},after=\vspace{1mm}]
            \item Algorithm Design: A
            \item Differential Eqns: A
            \item Linear Algebra: A
            \item Multivariable Calculus: A
            \item Computer Graphics: A
            \item Advanced Graphics: A
            \item Number Theory: A
            \item Theory of Computation: A
            \item Real Analysis: A
        \end{itemize}
        \vspace{-3mm}
    \end{multicols}
    \vspace{-3mm}
\end{rSection}
% \begin{rSection}{Career Interests}
%     \textbf{Teaching:}
%     I find particular enjoyment in working as a tutor, which motivates my
%     career interest of teaching. I would enjoy teaching at any level, but
%     teaching at the university level best aligns with my research interests.\\
%     \textbf{Further Education \& Research:}
%     I am interested in pursuing education in mathematics to the Ph.D level.
%     I have a particular interest in the mathematical branch of
%     differential geometry, which I would enjoy pursuing research in.
% \end{rSection}
\begin{rSection}{Research Experience}
\textbf{Identifying Clouds with Convolutional Neural Networks} \hfill
\textit{Summer 2017}\\
Our team worked towards an automated process to identify clouds in images
of the sky, which could assist climate scientists. We
implemented and trained a convolution neural network that can take a
picture of the sky as input and distinguish between clear, thin cloud, and
thick cloud pixels with 94\% accuracy.

\textbf{3-Orbifolds and their Laplace Spectra} \hfill \textit{Summer
2018}\\
This research project considers abstract geometrical constructions called
'orbifolds' (a generalization of a manifold) and asks: if it is only known
at what frequencies some unknown orbifold vibrates at (formally the Laplace
Spectra), what properties can we deduce about the orbifold? We found such a
property, which we call ``local orientability''.

\end{rSection}

\begin{rSection}{Academic Achievements} 
    \textbf{Published Academic Paper:} Primary author in the paper titled ``You can hear the local orientability of an orbifold'' in the Journal \textit{Differential Geometry and its Applications}\\
    \textbf{Acceptance to Pamplin Society:} Membership extended to seven
    students annually; ``the highest honor bestowed by the College on its
    students''.\\
    \textbf{Feynmann Book Award to Outstanding Introductory Physics Student:}
    An award bestowed by the physics faculty to the top stuent based on
    achievments in the first two years of physics courses.\\
    \textbf{CCSC Conference Presentation:} Poster presentation of cloud
    identification research in 2017 Consortium for Computing Sciences
    in Colleges Conference. Our poster won first prize.
\end{rSection}
\begin{rSection}{Other}
    \textbf{SQRC Tutor:} Tutor at the Lewis \& Clark Quantitative Resource
    Center for mathematics, physics, computer science. \textit{(Fall 2017-Present)}.\\
    \textbf{Grader:} A grader for an introductory physics class at Lewis \& Clark. \textit{(Fall 2019)}\\
    \textbf{Extra Curricular:} Four year varsity athlete in both the college
    cross country and track teams.\\
    \textbf{Programming Languages:} Fluency in C, C++, Java, Python,
    \LaTeX
\end{rSection}

% \begin{rSection}{References}
%     \begin{center}
%     \begin{tabular}{l l l}
%         Name         & \hspace{3cm} & Email \\
%     \hline
%     Iva Stavrov  & \hspace{3cm} & istavrov@lclark.edu \\
%     Liz Stanhope & \hspace{3cm} & stanhope@lclark.edu \\
%     Paul Allen   & \hspace{3cm} & ptallen@lclark.edu
%     \end{tabular}
%     \end{center}
% \end{rSection}
\end{document}
