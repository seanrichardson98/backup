\documentclass{resume} 
\usepackage[left=0.75in,top=0.6in,right=0.75in,bottom=0.6in]{geometry}
\usepackage{multicol}
\newcommand{\tab}[1]{\hspace{.2667\textwidth}\rlap{#1}}
\newcommand{\itab}[1]{\hspace{0em}\rlap{#1}}
\name{Sean Richardson}
\address{seanrichardson98@gmail.com}

\begin{document}
\vspace{-6.3mm}
\begin{rSection}{Education}
    \textbf{Lewis \& Clark College, Portland Oregon} \hfill \textit{September 2016 -
    Present}\\
    Cumulative GPA: 3.963,\ \ \ Math \& CS GPA: 4.0,\ \ \  Physics GPA:
    4.0\\
    Will graduate in \textit{May 2020} with:\\
    \vspace{-6.5mm}
    \begin{itemize}
        \setlength\itemsep{-2mm}
        \item Bachelor of Arts in Computer Science and Mathematics
        \item Bachelor of Arts in Physics
    \end{itemize}
    %\vspace{-2mm}
    %\textbf{Relevant Courses} will have taken before Summer 2019:
    \end{rSection}
\vspace{-1mm}
\begin{rSection}{Career Interests}
    \textbf{Further Education \& Research:}
    I am interested in pursuing education in mathematics to the Ph.D level.
    I have a particular interest in pursuing research in the branch of
    differential geometry\\
    \textbf{Teaching:}
    I find enjoyment in working as a tutor, which motivates my
    career interest of teaching. I would enjoy teaching at any level, but
    teaching at the university level aligns with my research interests.
\end{rSection}
\vspace{-1mm}
\begin{rSection}{Research Experience}
    \textbf{Identifying Clouds with Neural Networks} \hfill \textit{(Lewis \&
    Clark College)} \hfill
\textit{Summer 2017}\\
Our team worked towards an automated process to identify clouds in images
of the sky, which could assist climate scientists. We
implemented and trained a convolution neural network that takes a
picture of the sky as input and distinguishes between clear, thin cloud, and
thick cloud pixels with 94\% accuracy.

\textbf{3-Orbifolds and their Laplace Spectra} \hfill \textit{(Lewis \&
    Clark College)} \hfill \textit{Summer
2018}\\
In this project, we considered the abstract geometrical constructions of
`orbifolds' (generalizations of a manifold) and asked: if it is only known
at what frequencies an unknown orbifold vibrates at (formally the Laplace
spectra), what properties can we deduce about the orbifold? We found and
proved such a property, which we call ``local orientability''.
\end{rSection}
\vspace{-1mm}
\begin{rSection}{Academic Achievements} 
    \textbf{Acceptance to Pamplin Society:} Membership extended by Lewis \&
    Clark College to seven students annually; ``the highest honor bestowed
    by the College on its students''.\\
    \textbf{Feynman Book Award:} An award bestowed by the Lewis \& Clark
    physics department to the top student based on the first two years of
    work in the physics sequence.\\
    \textbf{Nomination by School for Goldwater Application:} One of four
    students per year selected by Lewis \& Clark College to apply for the
    Goldwater Scholarship.\\
    \textbf{CCSC Conference Presentation:} Poster presentation of cloud
    identification research in 2017 Consortium for Computing Sciences
    in Colleges Conference. Our poster won first prize.
\end{rSection}
\vspace{-1mm}
    \begin{rSection}{Relevant Courses}
    Will have taken the following before Summer 2019:
    \vspace{-1.5mm}
    \begin{itemize}
        \begin{minipage}{0.33\linewidth}
            \item Real Analysis I
            \item Real Analysis II
            \item Linear Algebra
        \end{minipage}
        \begin{minipage}{0.33\linewidth}
            \item Complex Variables
            \item Differential Equations 
            \item Intro to PDE's
        \end{minipage}
        \begin{minipage}{0.33\linewidth}
            \item Multivariable Calculus
            \item Theory of Computation
            \item Quantum Mechanics
        \end{minipage}
    \end{itemize}
\end{rSection}
\vspace{-1mm}
\begin{rSection}{Other}
    \textbf{SQRC Tutor:} Tutor at the Lewis \& Clark Quantitative Resource Center for
    mathematics, physics, computer science, logic, and basic economics.\\
    \textbf{Extra Curricular:} Athlete for the college cross country and
    track teams.\\
    \textbf{Programming Languages:} Fluency in C, C++, Java, Python,
    \LaTeX
\end{rSection}
\vspace{-1mm}
\begin{rSection}{References}

    Iva Stavrov: istavrov@lclark.edu $\vert$ Liz Stanhope:
    stanhope@lclark.edu $\vert$ Paul Allen: ptallen@lclark.edu\\

    %\begin{center}
    %\begin{tabular}{l l l}
    %    Name         & \hspace{3cm} & Email \\
    %\hline
    %Iva Stavrov  & \hspace{3cm} & istavrov@lclark.edu \\
    %Liz Stanhope & \hspace{3cm} & stanhope@lclark.edu \\
    %Paul Allen   & \hspace{3cm} & ptallen@lclark.edu
    %\end{tabular}
    %\end{center}
\end{rSection}
\end{document}
