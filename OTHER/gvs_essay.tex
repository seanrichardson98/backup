\documentclass[12pt]{amsart}                                                 
\usepackage[margin=1in]{geometry}

\DeclareMathOperator{\Div}{div}

\begin{document}
\title{Application Essay}
\author{Sean Richardson}
\maketitle

First, I will describe relevant research experience that has prepared me
for this REU program.\\

My research experience began in Summer of 2017 through the Lewis \& Clark
computer science department with a project on neural networks. Our team ---
research mentor Peter Drake, a fellow research student, and I --- worked
towards an automated process to identify clouds in images of the sky, which
could assist climate scientists. We implemented and trained a convolution
neural network that takes a picture of the sky as input and distinguishes
between clear, thin cloud, and thick cloud pixels with 94\% accuracy. This
was my first introduction to research and I enjoyed it, but over the next
year my interests became less focused on computer science and more focused
on pure mathematics.\\

So, in the Summer of 2018 I moved towards a pure mathematics research
project through Lewis \& Clark. In this project, my research mentor Liz
Stanhope and I considered the abstract geometrical constructions of
‘orbifolds’ (generalizations of a manifold) and asked: if it is only known
at what frequencies an unknown orbifold vibrates at (formally the Laplace
spectra), what properties can we deduce about the orbifold? We found such a
property, which we call ``local orientability''. We are currently in the
process of writing a manuscript on the proof of this result. Before this
project, I had never explored one specific area of mathematics so deeply,
but I enjoyed exploring the patterns that arise from the question and to
contribute a tiny additional piece of understanding to mathematics.\\

I will now address the Grand Valley State REU in particular. Firstly, I am
interested in the geometry research projects that I listed as my two
choices. I have some background in differential geometry from the orbifold
research project and from an informal independent study I do with professor
Iva Stavrov on Riemannian geometry and general relativity. Differential
geometry is the most fascinating branch of math to me so far, and I would
like to learn more. I would guess that the symplectic geometry topic would
provide a more broad understanding of differential geometry (being a topic
other than Riemannian geometry). And, I would guess that the spherical and
hyperbolic geometry topic would help build intuition for geometry studies.
So, both topics I would be interested in and gain useful skills from.
Additionally, it seems that the results of past programs have been
productive; I know I will find it exciting to not only be learning new
math, but contributing a small amount to math. On another note, it seems
that the program encourages participation in opportunities such as
mathematics conferences which is an opportunity I would certainly take and
be interested in. As a final note, my current career aspirations are to
teach and, if possible, continue in mathematics research. So, I would find
having a variety of research experiences useful to judge if I would like to
continue down this path.\\

Currently, I would say my favorite mathematical result would be Gauss's
divergence theorem for a variety of reasons. First of all, most of calculus
on manifolds is built on top of Gauss's theorem. For example,
we use Gauss's theorem to derive integration by parts which is a
fundamental maneuver for calculus of variations, and so on; calculus is
built on top of Gauss's theorem. A second reason, for I am majoring in
physics as well as math, is in the usefulness of Gauss's theorem in
electromagnetism and flux in physics. As an aside, don't worry: I use that
mathematical notation for divergence. A third reason, is a note on how
Gauss's theorem can be used to solve more classical mathematical problems.
A few months ago, I cam across the problem of finding the area of a
triangle with vertices at arbitrary points in space. One interesting way of
solving this problem is to create a simple vector field $V$ such that
$\Div(V)=1$. For instance, we can take $V = \langle x,0 \rangle$. Then, the
volume is simply the divergence integral over the region; then, by Gauss's
Theorem, this is simply the flux integral over the boundary. So, the
problem comes down to taking line integrals over straight lines --- a
fairly simple task. So, while there are other simpler methods of
solving this problem, Gauss's theorem can still be used to solve this
classical problem. 

\end{document}
