\documentclass[12]{amsart}

\usepackage{mathtools}                                                       
\usepackage{hyperref}                                                        
\usepackage{amsmath}                                                         
\usepackage{amssymb} 

\newcommand{\deriv}[2]{\frac{d#1}{d#2}}
\newcommand{\dderiv}[2]{\frac{d^2#1}{d#2^2}}
\newcommand{\abs}[1]{\vert#1\vert} 

\begin{document}
    \section{Fitting one resonance}
    Following the TeachSpin Quantum Analogs manual, we model the system close to a resonace frequency as a damped, driven, harmonic oscillator. Let $p$ denote the air pressure, let $\omega_0$ denote the specific resonance frequency, and let $\gamma$, and take $K$ and $\gamma$ to be constants corresponding to the power of the speaker and the damping of the system respecitively. Then, we have the linear differential equation
    \begin{equation*}
        \dderiv{p}{t} + 2\gamma\deriv{p}{t} + \omega_0^2p = K\cos(\omega t)
    \end{equation*}
    Then, we solve for the amplitude $A$ of the steady state solution, which will gives the long term behavior of the resonace. Combining the solution to the amplitude $A$ with the assumption $\omega_0 \gg \gamma$, this simplifies to the model.
    \begin{equation*}
        \abs{A(\omega)} = \frac{K}{\sqrt{(\omega_0-\omega)^2+\lambda^2}}
    \end{equation*}
For some variable $\lambda$. This gives a model for fitting a single peak. Combining this with the MATlab Curve Fitting Toolbox with free variables $K$, $\omega_0$, and $\lambda$, we can accuractely fit the given data such as in /**/.
    
    \end{document}
