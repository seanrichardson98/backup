\documentclass{article}[12pt]

\usepackage{amsmath}
\usepackage{mathtools}

\newcommand{\deriv}[2]{\frac{d#1}{d#2}}
\newcommand{\mymag}[1]{\vert\vert#1\vert\vert}


\begin{document}
    \section{Mathematical Model}

    We present a simple and accessible model to demonstrate the twisting
    phenomenon of the cat.

    Consider two identical rods $A$ and $B$, each with moment of inertia
    $I$, separated by angle $\theta$, and the whole structure maintains $0$
    angular momentum. The primary twisting movement of the
    cat requires $A$ and $B$ to /*twist*/. Assume that both rods $A$ and
    $B$ have angular velocity $\omega$ in their respective direction
    $\hat{a}$ and $\hat{b}$ as depicted in /**/. 

    Then the angular momentums $L_A$ and $L_B$ of the respective rods is
    given by
    \begin{align}
        \overrightarrow{L_A} = I\omega \hat{a} \text{\ \ \ and\ \ \ }
        \overrightarrow{L_B} = I\omega \hat{b} \text{\ \ \ and \ \ \ }
        \overrightarrow{L_C} = I_C\omega \hat{c}
        \label{eq:1}
    \end{align}

    However, we must have a counter rotation of the whole body by
    $\overrightarrow{L_C}$ such that angular
    momentum is conserved:

    \begin{equation}
        \overrightarrow{L_A} + \overrightarrow{L_B} + \overrightarrow{L_C}
        = \overrightarrow{0}
        \label{eq:2}
    \end{equation}

    We let $L_C = I_C \omega_C \hat{c}$ where $I_C$ is the moment of
    inertia at the center of mass about the $\hat{c}$ axis.

    Take unit vectors $\widehat{A}$ and $\widehat{B}$ as shown in /**/. We
    define the orientation of the cat as the direction of
    $\widehat{A}+\widehat{B}$. The orientation can vary by an angle $\phi$
    through a plane. We proceed to solve for $\deriv{\phi}{t} = \omega -
    \omega_c$.
    
    To solve for $\omega_C$, we simply combine the information listed in
    $(\ref{eq:1})$ and $(\ref{eq:2})$:
    \begin{align*}
        I\omega \hat{a} + I\omega \hat{b} + I_C\omega_C \hat{c} &=
        \overrightarrow{0}
    \end{align*}
    And solving for $\omega_c$,
    \begin{align*}
        \omega_C &= -\frac{I}{I_C}\omega \mymag{\hat{a}+\hat{b}} =
        -2\frac{I}{I_C}\omega \sin{\left(\frac{\theta}{2}\right)}\\
    \end{align*}
    Then overall,
    \begin{equation}
        \deriv{\phi}{t} =  
        \omega\left(1-2\frac{I}{I_C}\sin{\left(\frac{\theta}{2}\right)}\right)
    \end{equation}

\end{document}
