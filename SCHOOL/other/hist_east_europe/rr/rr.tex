\documentclass[12pt]{article}

\addtolength{\oddsidemargin}{-.875in}
\addtolength{\evensidemargin}{-.875in}
\addtolength{\textwidth}{1.75in}

\addtolength{\topmargin}{-.875in}
\addtolength{\textheight}{1.75in}

\begin{document}
\title{Reading Review}
\author{Sean Richardson}
\maketitle
\begin{itemize}
    \item The writer, Frantisek Palacky, thanks German Parliament for the invitation to the meeting at Frankfurt; the invitation reverses accusations that Palacky is an enemy of Germany. However, Palacky says he cannot come for the following reasons:
    \item Firstly, Palacky cannot support a meeting which aims to expand the power of the German Reich, for Palacky is not German. Further, if Palacky were to attend, his choices would be to mindlessly agree or rudely disagree.
    \item Palacky is Czech, and the Czech nation is independent with no obligation to the German Nation; the tie between Czechs and Germans is nothing more than a tie between the rulers of the nations.
    \item Palacky disagrees with the German Reich’s Aim to undermine Austria’s autonomy which is important to the well being of Europe.
    \item Russia is a vast power that is always looking to expand if they have the opportunity, which poses a threat to Slavs and Europe as a whole.
    \item In order to protect against Russia, Slavs must unite into a more cohesive unit.
    \item Many nationalities come together in Austria, giving Austria the ability and the responsibility to unify Slavs.
    \item Because Austria is the center of unification of Slavs, Palacky must disagree with the German Reich’s plans to undermine the power of Austria.
    \item Additionally, Palacky disagrees with the new form of government by “rule of the people” that the German Reich plans.
    \item Finally, Austria unification with Germany would be the end of Austria.
\end{itemize}


\end{document}
