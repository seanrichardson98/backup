\documentclass[12pt]{amsart}

\usepackage[margin=1in]{geometry}

\begin{document}
\author{Sean Richardson, Economic Development}
\title{Reading Review 2}
\date{\today}
\maketitle

In this reading review, I address Roland chapter 2 and touch on the
\textit{Poverty and Vulnerability Analysis} reading. Overall, Roland
Chapter 2 discusses different methods to measure poverty. Methods include
calorie counting --- but often nutrition is more complicated than calories.
We can also take use of calories to define a poverty line as the price of
the cheapest basket of goods that attains $2000$ calories and all those
under this income are considered poor. 
Roland discusses some individual formulas such as the headcount ratio, and
the poverty gap, but these are captured in the more general
Foster-Greer-Thorback expression (and it's variations) as discussed in DeJanvry and Sadoulet: 
$$P_\alpha = \frac{1}{n}\sum_i^q\left(\frac{z-y_i}{z}\right)^\alpha$$
where we take $n$ to be the total population, $z$ to be the poverty line,
$q$ to be the population under the poverty line, and $y_i$ to iterate over
the income of all those under the poverty line. We typically take $\alpha =
0,1,2$.
There is also the Lorentz Curve, which attempts to measure income
inequality. Each point on the Lorentz Curve denotes what percentage of
wealth the given percentage of poorest people control. This can be
simplified to an individual number, the Gini coefficient, which takes twice
the area between the line of equality and the Lorentz Curve.
Finally, it is important to keep in mind that to transition between various
currencies in this analysis, we make use of Purchasing Power Parity, which
is based on a typical basket of goods in two countries (not only traded
goods).\\

One observation I made had to do with how the Gini coefficient places equal
value for equal redistributions of wealth anywhere among the population.
Let me explain: if we take a wealth distribution of $\{1,2,3\}$ and
consider moving $1$ unit from person $1$ to person $2$ we arrive at
$\{0,3,3\}$ with a Gini coefficient of $0.33$. But, if we were to instead
move $1$ unit from person $2$ to person $3$ we arrive at the distribution
$\{1,1,4\}$ which has the same Gini coefficient of $0.33$. I believe this
holds in general and it would be interesting to mathematically prove this
conservation and more interesting to extend the proof of this behavior from
the discrete case to the continuous case. Additionally, I appreciated
DeJanvry and Sadoulet's discussion of how a higher value of $\alpha$ puts
more weight on the poorest person. As a policy maker this is important to
keep in mind: for $\alpha = 0$, it is beneficial to give aid to the richest
of the poor; for $\alpha = 1$, there is no bias to giving aid to any one of
the poor; and for $\alpha \geq 2$, it is beneficial to give aid to the poorest
of the poor.

\end{document}
