\documentclass[12pt]{amsart}

\usepackage[margin=1in]{geometry}
\newcommand{\pder}[2]{\frac{\partial#1}{\partial#2}}


\begin{document}
\title{Reading Review 4}
\author{Sean Richardson, Economic Development}
\maketitle

In this reading review I will discuss Roland Chapter 4. The chapter begins
in pointing out the economic growth of China and India. The importance of
looking economic growth in developing countries serves as a motivation for
the more technical contents. To begin the analysis of economic growth,
Roland first introduces factors of production --- primarily labor $L_t$ and
capital $K_t$. Then, the production function $F$ takes in the factors of
production as inputs and returns the value of outputs in the economy $Y_t$. A
specific example is the Cobb-Douglas production function:
\begin{equation}
    Y_t = A_t K_t^\alpha L_t^{1-\alpha} \ \ \ \ \ \ \ 0 \leq \alpha \leq 1
    \tag{Cobb-Douglas}
\end{equation}
The chapter than analyzes various aspects of this equation.

Firstly, I would like to point out the well known mathematical relation
(assuming continuous second derivatives)
\begin{equation*}
    \pder{}{L_t}\left(\pder{Y_t}{K_t}\right) = 
    \pder{}{K_t}\left(\pder{Y_t}{L_t}\right)
\end{equation*}
What we can take from this is that a small change in labor will affect the
marginal change in capital a similar amount that a small change in capital
will affect a marginal change in labor.
Additionally, I have one concern regarding the logic leading up to the
expression of the factor shares. The given formula for the labor shares
relies on the value of $\pder{F}{L_t}$ and $\pder{F}{K_t}$ reaching the
natural equilibrium of the wage rate and interest rate respectively,
maximizing return return. However, under Cobb-Douglas
equation with a constant return to scale variable $A_t$, I feel it is
important to address that it is mathematically possible to adjust the
inputs so that the partial derivatives can match these values for any of
the given values. In fact, because the expressions of both partial
derivatives are heavily reliant on the fraction $\frac{K_t}{L_t}$, I would
guess that this is not true for the Cobb-Douglas equation with constant
$A_t$. Overall, I appreciated this more mathematical approach to economics.
It seems as though the mathematics are well thought out and there is a lot
to learn from them.


\end{document}
