\documentclass[12pt]{amsart}

\setlength{\parindent}{0cm}
\usepackage[margin=0.8in]{geometry}

\usepackage{amsmath}
\usepackage{amsthm}
\usepackage{mathtools}

\usepackage{float}
\usepackage{enumitem}
\usepackage{commath}
\usepackage{tikz}
\usetikzlibrary{calc}
\usepackage{cancel}
\usepackage{subfigure}
\usepackage{multicol}
\usepackage{gensymb}

\usepackage{graphicx}
\graphicspath{{graphics/}}

\usepackage{my_notes}
\usepackage{my_math}

\theoremstyle{definition}
\newtheorem{theorem}{Theorem}[subsection]
\newtheorem{definition}[theorem]{Definition}
\newtheorem{lemma}[theorem]{Lemma}
\newtheorem{tool}[theorem]{Tool}
\newtheorem{approach}[theorem]{Approach}
\newtheorem{corollary}[theorem]{Corollary}

\newenvironment{just}{\textit{Justification.}}{}
\newenvironment{trans}{\textit{Translation.}}{}

\newenvironment{mybox}
{\begin{tcolorbox}[colback=red!5!white,colframe=red!75!black]}
{\end{tcolorbox}}

\newenvironment{mytbox}[1]
{\begin{tcolorbox}[colback=red!5!white,colframe=red!75!black,title=#1]}
{\end{tcolorbox}}

\def\mathunderline#1#2{\color{#1}\underline{{\color{black}#2}}\color{black}}

\begin{document}
\begin{theorem}
    The regular tetrahedron is not equidecomposable in the geometric sense
    to any rectangular box of the same volume.
\end{theorem}
\begin{proof}
    By problem 64, we have that $\alpha/\pi$ is irrational where $\alpha$
    denotes the dihedral angle of the regular tetrahedron. Then, by problem
    27, we can define an additive function $f$ such that $f(\alpha) = 1$
    and $f(\pi) = 0$. 
    So, we define a function $F$ as follows. For a polyhedron $P \subset
    \mathbb{R}^3$ with edge set $E$. For each edge $e \in E$, we denote the
    dihedral angle $\theta_e$ and the length $\abs{e}$.  Then, take $$ F(P)
    = \sum_{e \in E} \abs{e} f(\theta_e)$$ We claim $F$ is invariant
    between equidecomposable polyhedrons in the geometric sense. 

    Consider a polyhedron $P \subset \mathbb{R}^3$ with edge set $E$. Then,
    consider a planar cut of $P$ into two polyhedrons $P_1$ and $P_2$. For
    each $e \in E$, there are three cases. Either the planar cut (1)
    preserved the edge, (2) cut the edge into two edges of smaller length,
    preserving dihedral angle, or (3) cut the edge into two edges with
    smaller dihedral angle but preserving the length. \\
    
    In case 1, the value $\abs{e}f(\theta_e)$
    corresponding to $e$ is preserved, being part of either $F(P_1)$ or
    $F(P_2)$. 
    
    In case 2, $e$ is split into $e_1$ and $e_2$, but
    $$\abs{e}f(\theta_e) = (\abs{e_1}+\abs{e_2})f(\theta_e) =
    \abs{e_1}f(\theta_{e_1}) + \abs{e_2}f(\theta_{e_2})$$ And so the
    quantity $\abs{e}f(\theta_e)$ is preserved, being split between
    $F(P_1)$ and $F(P_2)$.

    In case 3, if $e$ is split into $e_1$ and $e_2$
    we have $$\abs{e}f(\theta_e) = \abs{e}f(\theta_{e_1}+\theta_{e_2}) =
    \abs{e_1}f(\theta_{e_1}) + \abs{e_2}f(\theta_{e_2})$$ by additivity of
    $f$. And so the quantity $\abs{e}f(\theta_e)$ is preserved, being split
    between $F(P_1)$ and $F(P_2)$. 
    
    Further, Each edge in either $P_1$ or $P_2$ results from one of these
    cases, so we can conclude that $F(P) = F(P_1) + F(P_2)$. We can extend
    this to conclude that $F(P) = \sum F(P_i)$ if $P = \bigcup P_i$ with a
    decomposition from planar cutting. But, if we have a decomposition
    $\{X_i\}$ of $P$ in general, then we can extend the planes of the
    polygons to make a refinement of planar cuts $\{Y_i\}$. Then, we by
    previous argument, $F(P) = \sum F(Y_i)$. And by gluing, we have $\sum
    F(Y_i) = \sum F(X_i)$. So, $F(P) = F(X_i)$ in general.

    Then, if we have two equidecomposable polygons $X = \bigcup X_i$ and $Y
    = \bigcup Y_i$ with $X_i, Y_i$ congruent. Then, $F(X_i) = F(Y_i)$ for
    each $i$ and so:
    $$F(X) = \sum F(X_i) = \sum F(Y_i) = F(Y)$$
    confirming that $F$ is an invariant over geometric equidecomposition.

    Finally, we have that the tetrahedron $T$ and any rectangle $R$ have
    different values under this invariant.
    $F(T) = 6(1\cdot f(\alpha))$ where $\alpha$ is the dihedral angle.
    Then, by definition of $f$, $F(T) = 6$.
    And, if $R$ has side lengths $l_1, l_2, l_3$, then $F(R) = 4 l_1
    f(\pi/2) + 4 l_2 f(\pi/2) + 4 l_3 f(\pi/2) = 0$ by definition of $f$
    and $f$ additive. 
\end{proof}
\end{document}

