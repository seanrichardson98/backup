\documentclass[12]{amsart}

%IMPORTANT PACKAGES:
\usepackage{mathtools}
\usepackage{hyperref}
\usepackage{amsmath}
\usepackage{amssymb}

\usepackage{enumerate}
\usepackage{xcolor}

\setlength{\parindent}{0em}
\setlength{\parskip}{.2em}

%NOTE
\newcommand{\toadd}[1]{/*#1*/}
\newcommand{\say}[1]{\textcolor{purple}{#1}}
\newcommand{\note}[1]{\textcolor{teal}{#1}}

\begin{document}
\title{Orbitope Presentation Outline}
\author{Sean Richardson}
\maketitle
\section{Intro}%
\toadd{give outline, some examples of orbitopes, etc.}

\section{Ingredients to Make an Orbitope}
\toadd{on small piece of paper so can carry it around}

INGREDIENTS:
\begin{enumerate}[(1)]
	\item A compact group $G$
	\item A real vector space $V$
	\item A group action acting linearly on the V.S $\rho: G \times V \to V$
	\item An element of the V.S.	$x$
\end{enumerate}

STEPS:

Consider the orbit $S$ of $x$ by $G$ with $\rho$. An orbitope is the convex hull of $S$.

\say{Don't need to be comfortable with everything now\dots}
\section{Example of a Diamond}
\begin{enumerate}[(1)]
	\item $G = Z_4 = <r>$
	\item $V = \mathbb{R}^2$
	\item $(r,v) \mapsto$ \toadd{rotation by $\pi/2$ counter clockwise}
	\item $x = (0,1)^T$
\end{enumerate}

\note{Show $S$ with picture}

\note{show orbitope with separate picture}
\section{Octagon Example?}
\toadd{todo}

\section{Octahedron Example}
\toadd{simplify construction}

\section{A Little Representation Theory?}

\section{Convex Geometry}
\toadd{definition of convex as smallest convex set}

\note{add picture}

\toadd{definition of convex hull}

\note{add picture}

\toadd{Show Intersection gives existence and uniqueness}

\section{Return to Definition of an Orbitope}

\section{Every Platonic Solid is an Orbitope}
\toadd{outline proof}

\section{Some More Convex Geometry}
\toadd{alternative definition of convex hull}

\toadd{Caratheodory's theorem and intuition}

\section{Application: Protein Folding Problem}
\subsection{Background and The Question}

\subsection{Measurements an What we Know}

\subsection{As a Mathematical Inverse Problem}

\subsection{Construction of Orbitope}

\subsection{What We Learn from the Orbitope}
\end{document}
