\documentclass[12]{amsart}
\begin{document}
\title{Project Proposal}
\author{Sean Richardson}
\date{\today}
\maketitle

    Both of my topics relate to Galois Theory; each topic is a theorem that
    uses Galois Theory in the proof. My first idea is to dig into the
    Abel-Ruffini Theorem --- the statement that there is no ``quintic
    formula''. I found a good document that builds towards this result with
    Galois theory, which is attached.  My reasoning behind Abel-Ruffini is that
    the statement itself is elementary, and the proof appears do-able at first
    glance. I think this is the topic I am most excited about. \\

    A separate proof I could focus on within Galois Theory would be Hilbert's
    Theorem 90. From the research I've done, the theorem seems like a big deal
    and I found a resource that builds towards the proof in an accessible and
    Galois Theory focused way, which is attached. The document is fairly long,
    but we covered field extensions last semester, so a substantial portion
    will be review. Although, I believe this is a more difficult project;
    simply understanding the precise statment of the theorem will require some
    work. \\

    One question I have is: what Galois Theory will we cover in class? What we
    see in class could affect the difficulty of these projects.
\end{document}
