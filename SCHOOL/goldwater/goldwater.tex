\documentclass{article}[11pt]

\usepackage{amsthm}
\theoremstyle{definition}
\newtheorem{theorem}{Theorem}
\newtheorem{definition}{Definition}

\begin{document}

\section{Research Essay}
\subsection{Abstract}
\begin{itemize}
    \item Can you hear the shape of a drum? (Intuition): Given a
        specific drum, a skilled mathematician could predict the precise
        sound the drum will make when hit. However, if the mathematician
        could only listen to some drum in the neighboring room, is it
        possible to reverse engineer the shape of the drum? Or could there
        exist multiple drums produce the same sounds? 
        
    \item (More Mathematical) This simple question sparked the field of
        \emph{Inverse Spectral Geometry}, which my research resides in.

    \item Sound is Mathematically formalized through a list called the
        \emph{Laplace Spectra}. This concept of sound then generalizes to
        not only drums and physical objects, but abstract mathematical
        constructs. In my research, we allowed the Laplace Spectra to
        extend to a class of objects called \emph{orbifolds}, which are
        motivated by symmetries.

    \item Question: In this research, we ask the question: ``Can you hear the
        shape of an orbifold?''. Or, given the Laplace Spectra of an
        orbifold

    \item Tie in historic motivation?
    \item Applications (Imaging --- detecting cracks in bridge support in
        Liz paper)
\end{itemize}
/*Give background as briefly as possible, to get into what we actually did*/
\subsection{Orbifolds}
This section addresses what an orbifold is:

An orbifold is a generalization of a manifold --- some multi-dimensional
surface. The local structure of manifold is restricted to euclidean space;
however, we allow 

\subsection{Laplace Spectra}

\subsubsection{Result}
The result of our research is the following new definition and theorem:
\begin{definition}[Locally non-orientable]
    We define the local structure of some orbifold to be
    $\emph{non-orientable}$ if the group action associated to the local
    structure contains a single orientation reversing element. We define an
    orbifold to be $\emph{locally non-orientable}$ if there exists any
    non-orientable local structure; otherwise, the orbifold is
    $\emph{locally orientable}$.
\end{definition}
\begin{theorem}
    No locally orientable orbifold can have the same Laplace Spectra as
    any locally non-orientable orbifold. In other words, we can hear the
    local orientability of an orbifold. 
\end{theorem}

\subsection{Methods}

\subsubsection{Asymptotic Expansion of the Heat Kernel}

\begin{itemize}
    \item list of coefficients $a,b,c,\dots$
\end{itemize}

\subsection{Reflection on Research?}
\subsection{Process of finding result}
\begin{itemize}
    \item No huge ``aha'' moment.
    \item From our computations, we noticed a pattern within the class of
        3-orbifolds.
\end{itemize}

\section{Questionnaire}

\subsection{Career Goals}
\subsubsection{Brief}

\subsubsection{Specific}
\begin{itemize}
    \item Differential Geometry?
    \item Mathematical Physics?
    \item Mathematical CS?
\end{itemize}
\subsubsection{Activity that helps enforce}
I.S?

\subsection{Research Activity}
\subsubsection{Cloud Research}
\subsubsection{Activities/Accomplishments}
\begin{itemize}
    \item xc/track
\end{itemize}
\subsubsection{}
\section{Questions}
Current goal: Explain what contributions I made and give the necessary
background to do this.
\begin{itemize}
    \item How much historical motivation?
    \item How much application?
    \item Reflection on research?
    \item A lot will need to go into background information even if I try
        to cut it down. Is this okay?
    \item Will probably go into research, but not sure what field. --- how
        apparent should I be with my uncertainty
\end{itemize}


\end{document}
