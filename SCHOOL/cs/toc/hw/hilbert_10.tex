\documentclass[12pt]{article}

\usepackage{amsmath}
\usepackage{amsthm}
\usepackage{amsfonts}

\usepackage{my_math}

\usepackage{sectsty}

\begin{document}

\title{\Huge{Hilbert's Problem}}
\author{\normalsize{Sean Richardson}}
\date{\normalsize{\today}}
\maketitle


\section{Algorithm}
We take input of a sequence of coefficients $\{a_0,a_1,\dots,a_n\}$
describing
a polynomial $p(x) = \sum_{k=0}^{n}a_k x^k = a_n x^n + a_{n-1}x^{n-1} + \cdots +
a_1 x^1 + a_0 x^0$ where the sequence continues until the highest nonzero
degree term in the polynomial which we call $n$.
\\We then have the following proposed algorithm:

\begin{enumerate}
    \item We define a bound $M \in \mathbb{Z}$ such than $M =
        \sum_{k=0}^{n-1}\myabs{a_k}$.
    \item Then, we iterate over the integers from $-M$ to $M$. For each
        instance $z \in \mathbb{Z}$ we have:
    \begin{enumerate}
        \item We evaluate $p(z)$. If $p(z) = 0$, we accept the input for
            $p$ has a integer root at input $z$.
            Otherwise, continue.
    \end{enumerate}
    \item If there are no roots within the interval we can safely reject,
        for we can guarantee that there are no roots outside of this
        interval. We have the proof below.
\end{enumerate}
\begin{proof}
    Take some $x \in \mathbb{Z}$ such that $\myabs{x} > M$. Then,
    $$ \myabs{a_n x_n} > \myabs{x^n} > \myabs{x^{n-1}}\myabs{x} >
    x^{n-1}\sum_{k=0}^{n-1}\myabs{a_k} $$
    Note $\myabs{x} \geq 1$ and thus $\myabs{x^k} < \myabs{x^{n-1}}$ for $k \leq n-1$.
    So, 
    $$\myabs{a_n x^n} > \sum_{k=0}^{n-1}\myabs{a_k x^{n-1}} >
    \sum_{k=0}^{n-1}\myabs{a_k x^k}$$
    For $a_n x^n > 0$ we then must have
    $$ a_n x^n > \sum_{k=0}^{n-1}\myabs{a_k x^k} > -\sum_{k=0}^{n-1}a_k x^k$$
    Thus we have 
    $$\sum_{k=0}^{n}a_k x^k = a_n x^n + \sum_{k=0}^{n-1} a_k x^k > 0$$
    Similarly, for $a_n x^n < 0$ we then must have
    $$ a_n x^n < -\sum_{k=0}^{n-1}a_k x^k$$
    Thus we have 
    $$\sum_{k=0}^{n}a_k x^k = a_n x^n + \sum_{k=0}^{n-1} a_k x^k > 0$$
    So, in both cases $p(x)$ cannot have a root, so it is safe to reject.
\end{proof}

\sectionfont{\fontsize{12}{15}\selectfont}
\section{Note}
I noticed that Hilbert's Problem is generalized for a multi-variable
equation. Perhaps this is a valid algorithm for one dimensional
polynomials, but does not work for more variables.

\end{document}
