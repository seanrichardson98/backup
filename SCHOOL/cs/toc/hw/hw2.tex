\documentclass{article}

\usepackage{amsmath}
\usepackage{amssymb}
\usepackage{amsthm}

\DeclareMathOperator{\floor}{floor}
\DeclareMathOperator{\range}{Range}
\DeclareMathOperator{\domain}{Domain}


\setlength{\parindent}{0cm}

\title{Homework 2}
\author{Sean Richardson}

\begin{document}

\maketitle

\subsection*{0.1}

\begin{itemize}
    \item[(e)] This is the set of all palindrome binary numbers
    \item[(f)] This is the empty set $\O$
\end{itemize}

\subsection*{0.6}

\begin{itemize}
    \item[(a)] $f(2) = 7$.
    \item[(b)] $\range(f)$ $ = \{ 6, 7 \}$ \\
    $\domain(f) = \{ n \mid n \in \mathbb{Z}, 1 \leq n \leq 5 \}$ 
    \item[(c)] $g(2,10) = 6$
    \item[(d)] $\domain(g) = \{ (p,q) \mid p,q \in \mathbb{Z}, 1 \leq p
        \leq 5, 6 \leq q \leq 10 \}$\\
        $\range(g) = \{ n \mid n \in \mathbb{Z}, 6 \leq n \leq 10 \}$
    \item[(e)] $g(4, f(4)) = g(4,7) = 8$
\end{itemize}

\subsection*{0.9}

$G = \{ V, E \}$ where $V$ is the set of vertices and $E$ is the set of
edges.

$$V = \{ 1, 2, 3, 4, 5, 6 \}$$

$$E = \{ (p,q) \mid \floor \left( \frac{p}{3} \right) = \floor \left(
    \frac{q}{3} \right) \}$$

\subsection*{0.10}

    The error in the proof is in the division by $(a-b)$ from both sides.
    Because $a=b$, this is division by $0$.

\subsection*{0.13}

Every graph with two or more nodes contains two nodes that have equal
degrees. I assume we are not allowing self loops, for we would have the
counter example $G = \{ \{ 1,2 \}, \{(1,1)\} \}$

\begin{proof}

We proceed by the method of induction.

We have the base case of $n=2$ in which the graph $G = \{\{0,1\}, E \}$ must
have $E = \O$ or $\{ (0,1) \}$. So, the base case holds.

We take the inductive hypothesis: that a graph of $n$ nodes must have two
nodes of the same order.

We will now make the inductive step: that under the inductive hypothesis, a
graph of $n+1$ nodes must have two nodes of the same order.

Consider a graph $G$ with $n+1$ vertices. Every node can connect to $n$
other nodes. In the case that  every node has at least order $1$, then each node can
have order $d$ such that $1 \leq d \leq n$. We have $n+1$ nodes and $n$
possible degrees, so two nodes must share the same degree. In the case that
there exists a node of order $0$, then we have a subgraph of order $n$
which must contain two nodes of the same degree by the inductive
hypothesis.

This demonstrates the inductive step and concludes our proof by induction.

\end{proof}

\end{document}
