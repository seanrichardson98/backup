%TODO: Shorten Research Experience section. Change title? Make links clearer? condense academic achievements?

\documentclass{resume} 
\usepackage[left=0.75in,top=0.6in,right=0.75in,bottom=0.6in]{geometry}
\usepackage{multicol}
\usepackage{enumitem}
\usepackage{hyperref}

\usepackage{enumitem}
\setlist{nosep}

\newcommand{\tab}[1]{\hspace{.2667\textwidth}\rlap{#1}}
\newcommand{\itab}[1]{\hspace{0em}\rlap{#1}}

\hypersetup{
    colorlinks=true,
    linkcolor=black,
    filecolor=magenta,      
    urlcolor=black,
}

\name{Sean Richardson}
\address{8622 SE 11th Avenue Portland, OR 97202}
\address{seanhrichardson@gmail.com \\ 503-707-9353 \\ seanhrichardson.com}

\begin{document}
\begin{rSection}{Education}
    \textbf{Lewis \& Clark College. Portland, Oregon (GPA: 3.973)} \hfill \textit{September 2016 - May 2020}\\
    Bachelor of Arts in Computer Science and Mathematics (GPA: 4.0)\\
    Bachelor of Arts in Physics (GPA: 4.0)
    %\textbf{Budapest Semesters in Mathematics. Budapest, Hungary (GPA: 4.0)} \hfill \textit{Summer 2019}
\end{rSection}

\begin{rSection}{Relevant Skills}
    \vspace{-3.5mm}
	\begin{multicols}{2}
        \begin{itemize}
        		\item Python experience from coursework.
            \item C and gdb experience from coursework.
            \item Java experience from coursework.
            \item Git user and experience using git with a team.
            \item Linux user and shell scripting experience.
            \item 30+ college level math/physics/CS classes.
        \end{itemize}
    \end{multicols}
      \vspace{-4mm} 
\end{rSection}

\begin{rSection}{Research Experience}
    \textbf{Mathematics Research in Differential Geometry} \hfill
    \textit{Lewis} \& \textit{Clark College, Summer 2018} 
    \vspace{-3.5mm}
    \begin{multicols}{2}
        \begin{itemize}
            \item Studied relevant literature and background.
            \item Conjectured and proved result.
            \item Created poster and powerpoint presentations.
            \item Wrote, edited, and published academic paper.
        \end{itemize}
    \end{multicols} 
    \vspace{-4mm}
    \textbf{Trained a Neural Network to Identify Cloud Images} \hfill
\textit{Lewis} \& \textit{Clark College, Summer 2017}
    \vspace{-3.5mm}
    \begin{multicols}{2}
        \begin{itemize}
        		\item Collaborated with Environ. Science Dept.
            \item Managed and preprocessed data.
            \item Reviewed and discussed relevant literature.
            \item Automated process with shell scripting.
        \end{itemize}
    \end{multicols} 
    \vspace{-4mm}
\end{rSection}

\begin{rSection}{Academic Achievements} 
    \textbf{Pi Mu Epsilon Honors Society Member} \hfill \textit{November 2019 --- Present}\\
 A national mathematics honors society; membership determined by college math faculty.

    %\href{https://www.sciencedirect.com/science/article/pii/S092622451930097X?utm_campaign=STMJ_75273_AUTH_SERV_PPUB&utm_medium=email&utm_dgroup=Email1Publishing&utm_acid=-800555120&SIS_ID=-1&dgcid=STMJ_75273_AUTH_SERV_PPUB&CMX_ID=&utm_in=DM597592&utm_source=AC_30 {
	\textbf{Published Academic Paper} \hfill \textit{October 2019} \\
Article titled ``You can hear the local orientability of an orbifold'' in the journal \textit{Differential Geometry and its Applications}. The paper presents a new theorem and proof.
%}

    \textbf{Scholar Athlete of the Year Award} \hfill \textit{May 2019}\\ 
An award bestowed by Lewis \& Clark athletic administration for academic and athletic achievements.

    \textbf{Feynmann Book Award to Outstanding Introductory Physics Student}
	\hfill \textit{May 2018}\\
    An award bestowed by college physics faculty to the top sophomore physics student based on academics.

    \textbf{CCSC Conference Poster Presentation First Prize} \hfill \textit{October 2017}\\
    Gave poster presentation and brief talk on summer 2017 research. Award decided by conference committee composed of faculty. CCSC Northwestern Region.

    \textbf{Pamplin Honors Society Member} \hfill \textit{September 2017 --- Present} \\
 Membership extended by the Society to seven Lewis \& Clark students annually for academics/leadership.

\end{rSection}

%\begin{rSection}{Leadership}
%
%    \textbf{Peer Tutor for Math, Physics, and Computer Science} \hfill \textit{September 2017 --- May 2020}\\
%    Taught and tutored material and encouraged peers. Occasionally led a class period in math.
%
%    \textbf{College Athlete in Cross Country and Track} \hfill \textit{September 2016 --- May 2020} \\
%    Gave younger athletes encouragement, advice, and a good example as the most experienced runner.
%
%\end{rSection}

\begin{rSection}{Other}
    \textbf{Grader for Math and Physics Classes} \hfill \textit{September 2019 --- May 2020}\\
    \textbf{Peer Tutor for Math, Physics, and CS} \hfill \textit{September 2017 --- May 2020}\\
    Taught and tutored material and encouraged peers. Occasionally led a class period in math.
\end{rSection}

\begin{rSection}{}
	More details on projects, course work, writing samples, and achievements at \textbf{seanhrichardson.com}
\end{rSection}

\end{document}

%Programming languages / technical skills section?

% \begin{rSection}{References}
%     \begin{center}
%     \begin{tabular}{l l l}
%         Name         & \hspace{3cm} & Email \\
%     \hline
%     Iva Stavrov  & \hspace{3cm} & istavrov@lclark.edu \\
%     Liz Stanhope & \hspace{3cm} & stanhope@lclark.edu \\
%     Paul Allen   & \hspace{3cm} & ptallen@lclark.edu
%     \end{tabular}
%     \end{center}
% \end{rSection}

%\begin{rSection}{Relevant Courses}
%    \vspace{-1mm}
%    \begin{multicols}{3}
%        \begin{itemize}[topsep=0pt,itemsep=0pt,parsep=0pt,before=\vspace{1cm},after=\vspace{1mm}]
%            \item Algorithm Design: A
%            \item Differential Eqns: A
%            \item Linear Algebra: A
%            \item Multivariable Calculus: A
%            \item Computer Graphics: A
%            \item Advanced Graphics: A
%            \item Number Theory: A
%            \item Theory of Computation: A
%            \item Real Analysis: A
%        \end{itemize}
%        \vspace{-3mm}
%    \end{multicols}
%    \vspace{-3mm}
%\end{rSection}


% \begin{rSection}{Career Interests}
%     \textbf{Teaching:}
%     I find particular enjoyment in working as a tutor, which motivates my
%     career interest of teaching. I would enjoy teaching at any level, but
%     teaching at the university level best aligns with my research interests.\\
%     \textbf{Further Education \& Research:}
%     I am interested in pursuing education in mathematics to the Ph.D level.
%     I have a particular interest in the mathematical branch of
%     differential geometry, which I would enjoy pursuing research in.
% \end{rSection}
%
%
%\begin{rSection}{Research Experience}
%\textbf{3-Orbifolds and their Laplace Spectra} \hfill \textit{Summer
%2018}\\
%This research project considers abstract geometrical constructions called
%'orbifolds' (a generalization of a manifold) and asks: if it is only known
%at what frequencies some unknown orbifold vibrates at (formally the Laplace
%Spectra), what properties can we deduce about the orbifold? We found such a
%property, which we call ``local orientability''.
%
%\textbf{Identifying Clouds with Convolutional Neural Networks} \hfill
%\textit{Summer 2017}\\
%Our team worked towards an automated process to identify clouds in images
%of the sky, which could assist climate scientists. We
%implemented and trained a convolution neural network that can take a
%picture of the sky as input and distinguish between clear, thin cloud, and
%thick cloud pixels with 94\% accuracy.
%
