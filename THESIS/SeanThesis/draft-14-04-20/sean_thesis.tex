%THE DOCUMENT
\documentclass[leqno]{amsbook}
\usepackage{geometry}
\geometry{letterpaper}

%MATH PACKAGES:
\usepackage{mathtools}
\usepackage{hyperref}
\usepackage{amsmath}
\usepackage{amssymb}
\usepackage{stmaryrd}

%ORGANIZATION PACKAGES:
\usepackage{subfiles}
\usepackage{import}
\usepackage{nameref,zref-xr}
\zxrsetup{toltxlabel}

%MISCELLANEOUS PACKAGES:
\usepackage{xcolor}
\usepackage{enumerate}
\usepackage{adjustbox}
\usepackage{subcaption}
%\usepackage[shortlabels]{enumitem}

%TIKZ:
\usepackage{tikz}
\usetikzlibrary{cd}

%NUMBERS:
\newcommand{\cdscale}{1.5}
\newcommand{\varparskip}{1em}

%THEOREM SETTINGS
\theoremstyle{definition}
\newtheorem{theorem}{Theorem}[chapter]
\newtheorem{definition}[theorem]{Definition}                                 
\newtheorem{claim}[theorem]{Claim}
\newtheorem{corollary}[theorem]{Corollary} 
\newtheorem{lemma}[theorem]{Lemma}   
\newtheorem{example}[theorem]{Example}

%SETTINGS
\setlength{\parindent}{0em}
\setlength{\parskip}{\varparskip}
%\setlist[itemize]{noitemsep, topsep=0pt}

%NOTE COMMANDS
\newcommand{\sean}[1]{\textcolor{purple}{#1}}
\newcommand{\iva}[1]{\textcolor{purple}{#1}}
\newcommand{\toadd}[1]{/*#1*/}

%MATH SHORTCUT COMMANDS
%justified shortcuts
\DeclareMathOperator{\vect}{Vect}
\DeclareMathOperator{\id}{Id}
\DeclareMathOperator{\im}{Im}
\newcommand{\iso}{\cong} %isomorphic symbol
\newcommand{\biso}{\approx}
\newcommand{\siso}{\approx_s} %stably isomorphic
\newcommand{\tiso}{\approx} 
\newcommand{\pset}[1]{\mathcal{P} (#1)} %power set
\newcommand{\KR}{\widetilde{K}} 
\newcommand{\set}[1]{\left\{#1\right\}}
\newcommand{\eset}{\emptyset}
\newcommand{\rest}[2]{{#1} \vert_{#2}}
\newcommand{\vecxx}[2]{#1 \choose #2}
\newcommand{\comp}[1]{\overline{#1}}
\newcommand{\abs}[1]{\vert #1 \vert}
%shortcuts for my extreme laziness
\newcommand{\T}{\mathcal{T}}
\newcommand{\N}{\mathbb{N}}
\newcommand{\C}{\mathbb{C}}
\newcommand{\R}{\mathbb{R}}
\newcommand{\Z}{\mathbb{Z}}
\newcommand{\U}{\mathcal{U}}
\newcommand{\F}{\mathcal{F}} 
\newcommand{\tbund}[1]{\varepsilon^{#1}}
\newcommand{\op}{\oplus} 
\newcommand{\ot}{\otimes} 
\newcommand{\RK}[1]{\widetilde{K}(#1)}

\begin{document}

\subfile{FrontMatter}

\tableofcontents

%\chapter*{Introduction}
%\sean{I may add a brief introductiony-ish thing that provides a brief summary (similar to what my slide presentation will be) of the direction that the book is going. It feels a bit harsh to push the reader into category theory and hope they survive until chapter 4 to finally see vector bundles... This  way I can also talk of the motivation to what I address along the way. I can use the word vector bundle earlier to give motivation and do a better job of giving a story.}

\subfile{CategoryTheoryChapter/CategoryTheoryChapter}

\subfile{LinearAlgebraChapter/LinearAlgebraChapter}

\subfile{TopologyChapter/TopologyChapter}

\subfile{VectorBundleChapter/VectorBundleChapter}

\subfile{KTheoryDefChapter/KTheoryDefChapter}

%\subfile{KTheoryCohomologyChapter/KTheoryCohomologyChapter}

%\subfile{DivisionAlgebraChapter/DivisionAlgebraChapter}

%\subfile{Appendix/Appendix}

\nocite{*}                                                                   
\bibliographystyle{plain}                                                    
\bibliography{bib}

\end{document}

