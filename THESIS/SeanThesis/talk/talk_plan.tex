%THE DOCUMENT
\documentclass[leqno]{amsart}

%MATH PACKAGES:
\usepackage{mathtools}
\usepackage{hyperref}
\usepackage{amsmath}
\usepackage{amssymb}

%ORGANIZATION PACKAGES:
\usepackage{subfiles}
\usepackage{import}
\usepackage{nameref,zref-xr}
\zxrsetup{toltxlabel}

%MISCELLANEOUS PACKAGES:
\usepackage{xcolor}
\usepackage{enumerate}
\usepackage{adjustbox}
\usepackage{subcaption}
%\usepackage[shortlabels]{enumitem}

%TIKZ:
\usepackage{tikz}
\usetikzlibrary{cd}

%NUMBERS:
\newcommand{\cdscale}{1.5}
\newcommand{\varparskip}{1em}

%THEOREM SETTINGS
\theoremstyle{definition}

%SETTINGS
\setlength{\parindent}{0em}
\setlength{\parskip}{\varparskip}
%\setlist[itemize]{noitemsep, topsep=0pt}

%NOTE COMMANDS
\newcommand{\sean}[1]{\textcolor{purple}{#1}}
\newcommand{\iva}[1]{\textcolor{purple}{#1}}
\newcommand{\toadd}[1]{/*#1*/}

%MATH SHORTCUT COMMANDS
%justified shortcuts
\DeclareMathOperator{\vect}{Vect}
\DeclareMathOperator{\id}{Id}
\newcommand{\iso}{\cong} %isomorphic symbol
\newcommand{\siso}{\approx_s} %stably isomorphic
\newcommand{\tiso}{\approx} 
\newcommand{\pset}[1]{\mathcal{P} (#1)} %power set
\newcommand{\KR}{\widetilde{K}} 
\newcommand{\set}[1]{\left\{#1\right\}}
\newcommand{\eset}{\emptyset}
\newcommand{\rest}[2]{{#1} \vert_{#2}}
\newcommand{\vecxx}[2]{#1 \choose #2}
%shortcuts for my extreme laziness
\newcommand{\T}{\mathcal{T}}
\newcommand{\N}{\mathbb{N}}
\newcommand{\C}{\mathbb{C}}
\newcommand{\R}{\mathbb{R}}
\newcommand{\Z}{\mathbb{Z}}
\newcommand{\U}{\mathcal{U}}
\newcommand{\F}{\mathcal{F}} 
\newcommand{\tbund}[1]{\varepsilon^{#1}}
\newcommand{\op}{\oplus} 
\newcommand{\ot}{\otimes} 
\newcommand{\RK}[1]{\widetilde{K}(#1)}

\begin{document}
	\title{Thesis Talk Outline}
	\author{Sean Richardson}
	\maketitle
	\begin{enumerate}
		\item Give intuitive introduction to vector bundles with lots of examples.
			\begin{enumerate}
				\item Tangent bundle to sphere
				\item Mobius Strip
				\item Trivial bundle definition
			\end{enumerate}
		\item Give necessary background to understand definition of V.B.
			\begin{enumerate}
				\item Topology motivation: Motivate the definition of open sets and continuous functions with visual $\R \to \R$ example.
				\item Give intuition for generalization with visual examples on sphere.
				\item Briefly address formal definition of topology and continuous functions.
			\end{enumerate}
		\item Give formal definition of a vector bundle, but emphasize a vector bundle as a bunch of fibers.
		\item Go over goal of the talk: given a topological space, want to translate the vector bundles over the topological space into a ring.
		\item Necessary algebra: 
			\begin{enumerate}
				\item \toadd{ring completion now? maybe later when there is better motivation for it}
				\item Motivation: need addition and multiplication due to V.B's being a union of fibers, turn to V.S's.
				\item Direct sum and tensor product on vector spaces with relevant properties.
			\end{enumerate}
		\item Extending direct sum and tensor product to V.B's by applying V.S. operations to each fiber.
			\begin{enumerate}
				\item Intuition for extending direct sum and tensor product
				\item Address identity elements for each operation
				\item Talk about $E \op E'$ trivial result. \toadd{but maybe later when there is more motivation}
			\end{enumerate}
		\item Go through operations of a ring and point out everything works except for additive inverses
		\item Another Algebra tangent:
			\begin{enumerate}
				\item What we have is a semiring \dots give definition of a semi ring.
				\item Example of a semiring: $\N \cup \set{0}$. Adding in the additive inverses gives $\Z$.
				\item This idea can be formalized so that every semiring has a unique ring extension so long as ... multiplication is commutative and ... there is the additive cancellation law
				\item Commutativity is good, but note that we have no tools that would promise an additive cancellation law
			\end{enumerate}
		\item Quest for getting cancellation property
			\begin{enumerate}
				\item However, here is a tool that could be of some use: ($E \op E'$ trivial theorem)
				\item Show some work trying to use this cancellation property:
				\begin{align*}
					&E \op F \iso E' \op F\\
					&E \op (F \op F') \iso E' \op (F \op F')\\
					&E \op \tbund{n} \iso E' \op \tbund{n}
				\end{align*}
				\item However all we get is (above). BUT now introduce very convenient equivalence relation: $E \siso E'$ if we have what is given above. Using this, we do have cancellation property! Adding in equivalence relation brings some complications with it (well-defined, etc.) but it all works out.
			\end{enumerate}
		\item Summary slide for the definition of K-Theory
	\end{enumerate}
\end{document}






