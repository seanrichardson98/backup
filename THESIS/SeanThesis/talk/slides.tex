\documentclass[12pt]{beamer}

\usetheme[progressbar=frametitle]{metropolis}
\usepackage{appendixnumberbeamer}
\usepackage{subcaption}
\usepackage{siunitx}

\title{Complex Topological K-Theory}
\date{\today}
\author{Sean Richardson '20}
\institute{Department of Mathematical Sciences, Lewis \& Clark College}

%NOTE COMMANDS
\newcommand{\sean}[1]{\textcolor{purple}{#1}}
\newcommand{\iva}[1]{\textcolor{purple}{#1}}
\newcommand{\toadd}[1]{/*#1*/}

%MATH SHORTCUT COMMANDS
%justified shortcuts
\DeclareMathOperator{\vect}{Vect}
\DeclareMathOperator{\id}{Id}
\newcommand{\iso}{\cong} %isomorphic symbol
\newcommand{\siso}{\approx_s} %stably isomorphic
\newcommand{\tiso}{\approx} 
\newcommand{\pset}[1]{\mathcal{P} (#1)} %power set
\newcommand{\KR}{\widetilde{K}} 
\newcommand{\set}[1]{\left\{#1\right\}}
\newcommand{\eset}{\emptyset}
\newcommand{\rest}[2]{{#1} \vert_{#2}}
\newcommand{\vecxx}[2]{#1 \choose #2}
%shortcuts for my extreme laziness
\newcommand{\T}{\mathcal{T}}
\newcommand{\N}{\mathbb{N}}
\newcommand{\C}{\mathbb{C}}
\newcommand{\R}{\mathbb{R}}
\newcommand{\Z}{\mathbb{Z}}
\newcommand{\U}{\mathcal{U}}
\newcommand{\F}{\mathcal{F}} 
\newcommand{\tbund}[1]{\varepsilon^{#1}}
\newcommand{\op}{\oplus} 
\newcommand{\ot}{\otimes} 
\newcommand{\RK}[1]{\widetilde{K}(#1)}

\begin{document}

\maketitle

\section{Introduction to Vector Bundles}

\begin{frame}{Vector Fields}
	\toadd{use vector fields in $\R^2$ and question of what do these vector fields live as motivation}
\end{frame}

\begin{frame}{Vector Fields on Sphere}
	\toadd{extend to tangent vector fields on a sphere. Break away from the vector fields and emphasize the object itself}
\end{frame}

\begin{frame}{The Cylinder}
	\toadd{Show construction of cylinder}
\end{frame}

\begin{frame}{The Mobius Band}
	\toadd{Show construction of Mobius Band}
\end{frame}

\begin{frame}{Trivial Bundle}
	\toadd{Emphasize cylinder as an example of a trivial bundle. Necessary for giving definition of vector bundle}
\end{frame}

\section{Topology Interlude}

\begin{frame}{Continuous Functions and Open Sets}
	\toadd{Give picture of a continuous and noncontinuous function $\R \to \R$ and emphasize the connection between continuity and open sets}
\end{frame}

\begin{frame}{Continuous Functions between Objects}
	\toadd{Generalize to a visual example of looking at the open sets and continuous functions between two topological spaces}
\end{frame}

\begin{frame}{Some Formal Topology}
	\toadd{Briefly give the formal definitions of a topological space and continuous functions in topology}
\end{frame}


\section{Back to Vector Bundles}

\begin{frame}{The Definition of a Vector Bundle}
	\toadd{Now ready to give a formal definition of V.B. Do this with some visual example as an aid}
\end{frame}

\begin{frame}{How to Think About Vector Bundles}
	\toadd{Emphasize what vector bundles are: topological objects but also a bunch of vector spaces shoved together. Introduce the vocabulary of fiber here}
\end{frame}

\begin{frame}{Homomorphisms on Vector Bundles}
	\toadd{Give definition of a homomorphism between vector bundles with an example as a visual aid}
\end{frame}

\begin{frame}{The Objective}
	\toadd{Formulate an objective for the talk: Take a topological object and use the vector bundles over that object to create a ring}
\end{frame}

\section{Algebra Interlude}

\begin{frame}{Definition of a Ring}
	\toadd{Give very brief definition of a ring for those unfamiliar}
\end{frame}

%MOTIVATION: Need operations on vector bundles. Because of the V.S. structure of V.B's, turn to V.S's.

\begin{frame}{The Direct Sum Operation on Vector Spaces}
	\toadd{Give some kind of definition of direct sum... will probably end up using the cartesian product}
\end{frame}

\begin{frame}{The Tensor Product Operation on Vector Spaces}
	\toadd{Give some kind of definition of tensor product. Honestly not sure how to do this so that it is accessible. I could say ``tensor product is a thing that gives distribution over direct sum'' and leave it there...?}
\end{frame}

\begin{frame}{Properties of Direct Sum and Tensor Product}
	\toadd{Give list of nice properties of tensor product an direct sum (associativity, commutativity, distributativity...)}
\end{frame}

\section{Extending Vector Space Operations to Vector Bundles}

\begin{frame}{Extending Direct Sum to Vector Bundles}
	\toadd{Emphasize that a V.B. is a simply a bunch of fibers, so can simply apply the direct sum operation to each fiber}
	\toadd{Mention that there is more to do (give a topology and check local triviality, but it all works out)}
\end{frame}

\begin{frame}{Example of Direct Sum of Vector Bundles}
	\toadd{Some sort of example of direct sum... not sure what a good choice here would be}
\end{frame}

\begin{frame}{Extending Tensor Product to Vector Bundles}
	\toadd{Say that tensor product is basically the same thing as direct sum ... apply the T.P. operation to each fiber}
\end{frame}

\begin{frame}{The Ring Properties of Vector Bundles}
	\toadd{side by side comparison of properties of vector bundles and properties of a ring. Emphasize lack of additive identity}
\end{frame}

\section{Another Algebra Interlude}

\begin{frame}{Semirings}
	\toadd{def of semiring, example of $\N \cup \set{0}$}
\end{frame}

\begin{frame}{Ring Extension}
	\toadd{$\N \cup \set{0}$ extends to $\Z$. This generalizes given ... commutativity of multiplication and additive cancellation law}
\end{frame}

\begin{frame}{The Semiring Properties of Vector Bundles}
	\toadd{side by side comparison of properties of vector bundles and properties of ring. Emphasize lack of cancellation law}
\end{frame}

\section{Quest for the Additive Cancellation Property}

\begin{frame}{Useful Tool}
	\toadd{Mention that given certain constraints, for any bundle $E$ there exists a bundle $E'$ such that $E \op E'$ is trivial}
	\toadd{Include visual example of this... the best visual example might be the normal and tangent bundles on a sphere}
\end{frame}

\begin{frame}{An Attempt at Using the Tool}
	\toadd{
		Show some work trying to use this as a cancellation property:
		\begin{align*}
			E \op F &\iso E' \op F\\
			E \op (F \op F') &\iso E' \op (F \op F')\\
			E \op \tbund{n} &\iso E' \op \tbund{n}
		\end{align*}
	}
\end{frame}

\begin{frame}{A Convenient Equivalence Relation}
	\toadd{
	the work on the previous slide motivates the equivalence relation $E \siso E'$ if $E \op \tbund{n} \iso E' \op \tbund{n}$ for some $n$.
	}
	\toadd{mention the complications of introducing an equivalence relation (well-defined) but it all works out}

\end{frame}

\section{The Definition of K-Theory}

\begin{frame}{K-Theory}
	\toadd{A summary slide that gives the definition of K-Theory}
\end{frame}

\begin{frame}{???}
	\toadd{ideas on how to expand:
		\begin{itemize}
			\item Incorporate category theory and talk about K-Theory as a functor. Would require introducing pullback bundles and putting more emphasis on homomorphisms.
			\item Introduce cohomology theory and explain how K-Theory extends to a cohomology theory (would probably require the above)
			\item Work towards the division algebra  application. A semi-manageable goal would be the proof showing odd dimensions are impossible (would probably require the above)
		\end{itemize}
	}
\end{frame}

\end{document}
