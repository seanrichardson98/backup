\documentclass[12]{amsart}

%IMPORTANT PACKAGES:
\usepackage{mathtools}
\usepackage{hyperref}
\usepackage{amsmath}
\usepackage{amssymb}

%RANDOM PACKAGES:
\usepackage{xcolor}
\usepackage{enumerate}

%THEOREM SETTINGS
\newtheorem{theorem}{Theorem}
\newtheorem{definition}[theorem]{Definition}                                 
\newtheorem{claim}[theorem]{Claim}
\newtheorem{corollary}[theorem]{Corollary}  
\newtheorem{example}[theorem]{Example}

%NOTE COMMANDS
\newcommand{\sean}[1]{\textcolor{purple}{#1}}
\newcommand{\iva}[1]{\textcolor{purple}{#1}}
\newcommand{\toadd}[1]{/*#1*/}

%MATH SHORTCUT COMMANDS
%justified shortcuts
\DeclareMathOperator{\vect}{Vect}
\DeclareMathOperator{\id}{Id}
\newcommand{\iso}{\cong} %isomorphic symbol
\newcommand{\siso}{\approx_s} %stably isomorphic
\newcommand{\tiso}{\approx} 
\newcommand{\pset}[1]{\mathcal{P} (#1)} %power set
\newcommand{\RK}[1]{\widetilde{K}(#1)} 
\newcommand{\KR}{\widetilde{K}}
\newcommand{\set}[1]{\{#1\}}
%shortcuts for my extreme laziness
\newcommand{\T}{\mathcal{T}}
\newcommand{\N}{\mathbb{N}}
\newcommand{\C}{\mathbb{C}}
\newcommand{\R}{\mathbb{R}}
\newcommand{\Z}{\mathbb{Z}}
\newcommand{\U}{\mathcal{U}} 
\newcommand{\tbund}[1]{\varepsilon^{#1}}
\newcommand{\op}{\oplus} 
\newcommand{\ot}{\otimes} 

%LIST:
\newcommand{\itemc}{\item[\checkmark]}
\newcommand{\itemp}{\item[$\sim$]}
\newcommand{\itemo}{\item[$\circ$]}
\newcommand{\items}{\item[$\circ$]}

\begin{document}
\title{Thesis Outline}
\maketitle

\section{Category Theory}%CATEGORY THEORY:\\
\begin{itemize}
    \itemc Definition of Category
    \itemc Definition of Functor
    \itemc Category of Rings and/or Groups
    \itemo Category of sets
    \itemc Category of Vector Spaces 
    \itemc Definition of isomorphism
    \itemo Category of abelian group sequence?
    \itemo Category of pairs?

\section{Algebra}%LINEAR ALGEBRA:
\subsection{Ring Completion}~
	\itemc Ring completion definition
	\itemc Ring completion verification
	\itemc Ring completion Example 
    \itemo (for n-points): ring completion of $(\N \cup \set{0})^n$
\subsection{Packing together Modules}~
    \itemc Direct sum on modules definition
    \item Verifications for direct sum
    \item Extension to other categories
    \item Example for vector spaces: $\R^n \op \R^m \iso \R^{n+m}$
    \itemc Tensor product on modules definition
	\item Verifications for tensor product
    \item Extension to other categories 
    \item Example for vector spaces: $\R^n \ot \R^m \iso \R^{nm}$
    \itemo Example/Result: $\Z[\alpha]/(\alpha^2)\ot\Z[\beta]/(\beta^2) \iso 
    		   \Z[\alpha,\beta]/(\alpha^2,\beta^2)$  
    \itemo Idea of a \emph{outer product} i.e. \emph{external product} to make $\mu$ less jarring later.  

\section{Topology}%TOPOLOGY
\subsection{Basic Definitions}~
    \itemc Definition of topological space
    \itemc Definition of HM. i.e. cnts function
\subsection{Category of Topological Spaces}~
	\itemc Definition of the Category
	\itemc Formal comparison to metric space with functor
	\itemc Construction of $S^2$ as example	
\subsection{Mappings in Topology}~
	\itemc Definition of quotient map and quotient topology
	\itemo Example: $I$ to $S^1$
	\itemc Definition of inclusion map and subspace topology
	\itemo Example of $\R P^1$
	\item Definition of $\C P^1$
	\item Theorem: $\C P^1 \iso S^2$
\subsection{Operations on Topological Spaces}~
	\item Wedge Sum
    \item Smash Product
    \item Cone Definition
    \item Example: $D^n$ construction
    \item Suspension Definition
    \item Example: $S^n$ construction
    \item Reduced Suspension
    \item Relevant sequence construction.
    \item Discuss the ``union'' sign.
\subsection{More Topology}~
    \itemc Definition of Compact
    \itemc Definition of Hausdorff
    \itemo Definition of Normal
    \itemo Theorem: Every Compact Hausdorff Space is Normal 
\subsection{Homotopy Things}~
	\item Homotopic Functions
	\item Homotopic Spaces
    \item Contractible definition
    \item Examples: those necessary for ch. 6 costruction
\subsection{Appendix}~
    \itemo Urysohn Lemma
\section{Vector Bundles} %VECTOR BUNDLES
\subsection{Basic Definition and Examples}~
	\itemp Motivation: vector fields
    \itemc Definition of V.B.
    \itemc Brief definition of fiber.
    \itemo Brief definition of section.
    \item Example: Definition of trivial bundle of dimension $n$: $\varepsilon^n$.
    \item Example: Tangent Bundle over $S^1$ or $S^2$.
    \item Example: Mobius band
    \itemo Canonical line bundle over $\R P^1$ gives mobius band
\subsection{Category Theory of Vector Bundles}~
    \itemc Definitions of homomorphism, isomorphism. 
    \itemc Explanation of Category
    \item Definition of restriction
    \item Vereification that restrction is a vector bundle.
    \itemc Definition of $\oplus$ on vector bundles
    \itemc Properties of $\oplus$ on vector bundles (required for K-Theory def)
    \itemc Verifications for $\oplus$
    \itemc Definition of $\otimes$ on vector bundles.
    \itemc Propertues of $\oplus$ on vector bundles (required for K-Theory def)
    \itemc Verifications for $\otimes$
    \itemc Definition of pullback bundle
    \itemc Verifications for pullback bundles
   	\itemc Important properties of pullback bundles
    \itemc Verification of properties of pullbacks
    \item Note restriction as example of pullback
\subsection{Necessary Results on Vector Bundles}~
	\item Theorem: If $H$ is canonical line bundle on $\C P^1$, $(H \ot H) \op 1 \iso H \op H$
	\itemo Verification for above.
	\itemc Theorem: For every vector bundle $E$, there exists an $E'$ such that $E \op E'$ is trivial.
	\itemo Verification for above.
	\itemo Example of above. Mobius band with itself.
    \item Pullback from homotopic map gives an isomorphism.
	\item Corrolarry: $A$ contractible implies the bundle over $A$ is trivial.
\section{Definition of K-Theory} %K-THEORY:
\subsection{The K-Theory Functor $K$}~
    \itemc Definition of $K$ functor on topological spaces.
    \itemc Verification that above gives ring.
    \itemc Definition of $K$ functor on homomorphisms.
    \itemc Verification that above gives homomorphism of rings.
    \itemc Verification that above is a functor.
    \item Simple Examples
\subsection{The Reduced K-Theory Functor $\KR$}~
    \item Definition of $\KR$ functor on topological spaces.
    \item Verification that above gives ring.
    \item Definition of $\KR$ functor on homomorphisms.
    \item Verification that above gives homomorphism of rings.
    \item Verification that above is a functor.
    \item Simple Examples
	\item Theorem: $K(X) \iso \KR(X) \op \Z$
\subsection{Notation}~
    \item Describing elements of $K$ and $\KR$
    \item pullback notation


\section{K-Theory as a cohomomology theory}~
\subsection{Exact Sequences}~
    \item Short exact sequence definition
    \item State splitting Lemma
    \itemo Verify splitting lemma
    \item Verify $\KR$ and $K$ relations claimed previously
    \item Discuss that $(X,A)$ induces a short exact sequence
    \item Verify that the above is indeed a short exact sequence
    \item Extend above sequence to a long sequence
    \item Talk about homotopy equivalences and K-Theory at some point.

\subsection{Extending to a cohomology theory}~
    \item Discussion of what constitues a cohomology theory and why the reader should care?
    \item The construction

\subsection{Patterns of K-Theory}~
    \item Definition of the external product
	\item The external product $\KR(S^{2k}) \ot \KR(X) \to \KR(S^{2k} \wedge X)$ is an isomorphism.
    \item Bott Periodicity\dots
    \item $\KR(S^n)$ is $\Z$ for $n$ even and $0$ for $n$ odd.
   	\item $K(S^{2k}) \ot K(X) \to K(S^{2k} \times X)$ is an isomorphism.

\section{Division Algebra Application}~
\subsection{Division Algebras and Paralizable Spheres}~
    \item Define division algebras
    \item reduce division algera problem to a K-Theory problem
    \itemo Define paralizable spheres
    \itemo reduce paralizable spheres to a K-Theory problem
\subsection{Even Case}~

\subsection{Odd Case}~
        
        \section{Overall Themes and Ideas}~
    \item Tell the reader right at the beginning the plan to get a ring out of a space and remind the reader of this goal all the way through.
    \item I may add a brief introductiony-ish thing that provides a brief summary (similar to what my slide presentation will be) of the direction that the book is going. It feels a bit harsh to push the reader into category theory and hope they survive until chapter 4 to finally see vector bundles... This way I can also talk of the motivation to what I address along the way. I can use the word vector bundle earlier to give motivation and do a better job of giving a story.
\end{itemize}
\end{document}
