\documentclass[12]{amsart}
\newcommand{\op}[0]{\oplus}
\newcommand{\ot}[0]{\otimes}
\newcommand{\RK}[0]{\tilde{K}}
\newcommand{\Z}[0]{\mathbb{Z}}
\newcommand{\ismt}[0]{\approx} 
\begin{document}
\title{Reading Questions 18 Feb 2020}
\maketitle

Hi Iva! I re-read the bottom of 53, and read pages 54, 55, and some of 56.

\begin{enumerate}
    \item After some googling, I learned that over groups, the direct sum $\op$ and the tensor products $\ot$ distribute over groups and further it seems that $\Z$ acts as some kind of ``multiplicative identity'' for the tensor product over groups. This seems to be used multiple times throughout page 54. For instance, I'm pretty sure that the $K(X) \ot K(Y) \ismt (\RK(X) \ot \RK(Y)) \op \RK(X) \op \RK(Y) \op \Z$ result in the middle of 54 requires using things along these lines. Should I know more about this?
    \item More googling led me to find this thing called the ``Splitting Lemma'', which I think Hatcher assumes I know. It seems this lemma is used multiple times at the bottom of pae 53 and throughout page 54. Is Hatcher using this? If so, should I know more about this?
    \item Also, Hatcher mentions $\RK(S^2)$ is infinite cyclic, which is a very important part of the proof for Bott Periodicity and is used later. I am not sure why this is true. At a glance, it seems likely that this comes from somewhere in the 42 --- 51 page range that Hatcher promised me I could skip. So, perhaps I'll have to dig into that section eventually after all.
    \item On page 55, there is the phrase ``coming from the products $\RK(\Sigma^i(X/A)) \ot \RK(\Sigma^j(Y/B)) \rightarrow \RK(\Sigma^{i+j}(X/A \land Y/B))$. However, I am not sure how this ties in with the ``relative form of the produt'' --- isn't this ``relative form'' simply come from a change of variables?
\end{enumerate}

Other than these questions, I atually think I understood the reading quite well.
\end{document}
