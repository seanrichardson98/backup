\documentclass[12]{amsart}

\begin{document}

\title{Reading Questions 02/11}
\maketitle

Hi Iva, I read the Hatcher notes (attached) pages 40, 41, and the start of 42. I skipped 42 -- 51, which is all the proof of theorem 2.2 (this theorem is the hard part of priving Bott Periodicity Theorem I believe), then I read starting from Section 2.2 pages 51, 52, 53, and starting 54. Here are my questions:\\

\begin{enumerate}
    \item At the top of 41, the function $\mu: K(X) \otimes K(Y) \to K(X \times Y)$. Firstly, I don't think I'm familiar with the tensor product between groups; I imagine it is pretty similar to the stuff we did in Real 2, but should I know more? Secondly, this map is called an \emph{external product}; is that the name of this particular map or is it a technical term I should know?\\

    \item At the bottom of 52, the long chain of mappings $A \to X \to X \cup CA \to \dots$. What does $X \cup CA$ technically mean? Where exactly are these two spaces attached? My guess would be that if $CA = (A \times I)/(X \times \{0\})$, these two spaces share the ``$0$'' part of the interval. But perhaps it is a disjoint union; not sure.\\

    \item On 53, the chain of mappings $\dots \to \tilde{K}(SX) \to \tilde{K}(SA) \to \tilde{K}(X/A) \to etc$. How does do we get this? It is certainly a combination of Lemma 2.10 and the chain from my previous question. Perhaps the vertical maps from the chain of mappings on 53 can be expressed by some ``obvious'' quotients, which allows us to apply Lemma 2.10. Unfortunately, I have no clue what these quotient maps would be.\\

    \item On 53, the short paragraph immediatley after the chain of mappings sarting with ``for example''. What does it mean for the sequence to break up into short exact sequences and how is that true in this case? This gives us the result $\tilde{K}(A \lor B) \simeq \tilde{K}(A) \oplus \tilde{K}(B)$, which is used later so this seems important to know.\\

    \item Hatcher often references $\mathbb{C}P^1$ and he tells me that this is the same thing as $S^2$. He provided a proof in chapter 1 for this that mostly convinced me, but I think I am still not as comfortable with $\mathbb{C}P^1$ and visualizing it as I should be.\\

\end{enumerate}

Here are some questions that I had, but I believe I have the answer to. I am mostly including these for my own reference, but perhaps you have some insights.

\begin{enumerate}
    \item What is ``$\approx$''? A: Apparently another way to write isomorphic.\\

    \item Why does $X$ compact Hausdorff and $A \subset X$ imply $X/A$ compact Hausdorff? A: I believe I see the proof. Compact is easy and Hausdorff is easy with the $X$ compact Hausdorff $\implies$ $X$ normal result.\\

    \item On 53, why does $A$ contractible imply bundle over $A$ trivial? A: This proof is in Attiyah, and given the amount of time I spent trying to understand that proof I can't believed I forgot that result.


\end{enumerate}
\end{document}
