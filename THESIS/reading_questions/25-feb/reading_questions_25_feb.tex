\documentclass[12]{amsart}

\usepackage{mathtools}
\usepackage{hyperref}
\usepackage{amsmath}
\usepackage{amssymb}

\newcommand{\RK}{\widetilde{K}} 
\newcommand{\tiso}{\approx} 
\newcommand{\Z}{\mathbb{Z}}

\title{Reading Questions 25 February}

\begin{document}
\maketitle
Hi Iva! I made another pass through page 55 and a first pass through page 56 (and a tiny bit of 57).  I also began reading the division algebras section in parallel (I think I am eager to get to the punch line, is reading this part now a good idea? Not sure.) So, I also read page 59 and 60. Here are my questions:

\begin{enumerate}	

	\item On page 60 under item number (3), Hatcher mentions that $K(S^{2k}) \tiso \Z[\alpha]/(\alpha^2)$, and I am not sure why this is true. I believe theorem 2.2 on page 41 implies $K(S^{2k}) \tiso \Z[H]/(H-1)^2$. Is it true that $\Z[\alpha]/(\alpha^2) \tiso \Z[H]/(H-1)^2$ then? I would buy it, for isn't $\Z[x] = \X[x+1]$, so it is kind of just a change of variables?
	
		\item At the bottom of page 56, Hatcher suddenly introduces ``${\RK}^*(X)$--modules''. I suppose I now know what modules are, so that is good. I understand ``${\RK}^*(X)$--modules'' to mean that ${\RK}^*(X)$ is the scalar ring that is now associated to the vector space. But, it seems that these ``vector spaces'' are ${\RK}^*(A)$ and ${\RK}^*(X,A)$, which seems strange. The top of page 57 explains how this is possible which I buy, but this simply feels like a strange construction. What is the purpose of introducing modules here? Perhaps I simply need to read more and answer this for myself.
	
	\item Not really a question, but I am still dijesting the proof of proposition 2.14. I believe I can understand it by myself, but perhaps it is something we could go over if we have the time.
	
\end{enumerate}

	That's it! It seems my questions are kind of minor this week. I wonder if that is a good or bad sign. Not sure.


\end{document}
