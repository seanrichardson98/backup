\documentclass[12]{amsart}

\usepackage{mathtools}
\usepackage{hyperref}
\usepackage{amsmath}
\usepackage{amssymb}

\newcommand{\RK}{\widetilde{K}} 
\newcommand{\tiso}{\approx} 
\newcommand{\Z}{\mathbb{Z}}

\title{Reading Questions 4 March}

\begin{document}
\maketitle

Hi Iva! This week, I primarily re-read the stuff on page 61, working through the definition of $\hat{g}$ for an explicit example in the $n=1$ case, and I believe I now understand the proof of Lemma 2.18. I also attempted to understand the discussion on page 63, which motivates the construction used in the proof of Theorem 2.20, but I certainly have more work to put in before I totally understand that; it is less intimidating now though. Here are my questions:

\begin{enumerate}
	\item Directly after the statement of Theorem 2.20, Hatcher says that ``properties (2) and (3) give $\psi^{k}(L_1 \oplus \dots \oplus L_n) = L_1^k + \dots + L_n^k$.'' I do not understand this yet, and I feel that if I have any hope of understanding the next two pages, I should certainly understand this. Here are my thoughts/questions so far on this:
	\begin{enumerate}
		\item Firstly, what does the notation $L^k$ mean? I would guess this $k$ tensor products of $L$?
		\item Also, when we apply $\psi(L)$, we are actually acting on the equivalence class of $L$.
	\end{enumerate}
	Overall, I'm not sure how to get this addition result using (2) and (3).
	\item Secondly, at the bottom of page 60 in the list of four properties on exterior powers, I do not think I am familiar with $(i)$. I buy all of the other properties though.
	\item Also, not a question but I reread some of the earlier section of Hatcher and figured out why we have the cancellation property for the monoid. I believe it comes from Proposition 1.4. Also, as far as I can tell, I believe all of the Compact Hausdorff assumptions stem from this Proposition, for it uses Urysohn’s Lemma. Interesting! 
\end{enumerate}
\end{document}