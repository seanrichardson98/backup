\documentclass{amsart}
\usepackage{setspace, amssymb, amsmath, amsfonts}
\usepackage{proofread}


\theoremstyle{plain}
\newtheorem{thm}{Theorem}[section]
\newtheorem{theorem}[thm]{Theorem}
\newtheorem{cor}[thm]{Corollary}
\newtheorem{lem}[thm]{Lemma}
\newtheorem{lemma}[thm]{Lemma}
\newtheorem{prop}[thm]{Proposition}
\newtheorem{proposition}{Proposition}

\theoremstyle{definition}
\newtheorem{defn}[thm]{Definition}
\newtheorem{definition}[thm]{Definition}
\newtheorem{ex}[thm]{Example}
\newtheorem{example}[thm]{Example}
\newtheorem{examples}[thm]{Examples}

\theoremstyle{remark}
\newtheorem{remark}[thm]{Remark}
\newtheorem{remarks}[thm]{Remarks}
\newtheorem{notation}[thm]{Notation}




%New Commands

\newcommand{\myabs}[1]{\vert#1\vert}

\newcommand{\ld}{\lambda}
\newcommand{\cp}{\mathcal{P}}
\newcommand{\mfh}{\mathfrak{h}}
\newcommand{\mfk}{\mathfrak{k}}
\newcommand{\mfp}{\mathfrak{p}}
\newcommand{\mfv}{\mathfrak{v}}
\newcommand{\mfz}{\mathfrak{z}}
\newcommand{\bv}{\boldsymbol{v}}
\newcommand{\bx}{\boldsymbol{x}}
\newcommand{\C}{\mathbb{C}}
\newcommand{\bbH}{\mathbb{H}}
\newcommand{\R}{\mathbb{R}}
\newcommand{\Z}{\mathbb{Z}}
\newcommand{\cpt}{\mathbb{CP}^2}
\newcommand{\pa}{\partial}
\newcommand{\wtu}{\widetilde{U}}
\newcommand{\wtn}{\widetilde{N}}
\DeclareMathOperator{\Id}{Id}
\newcommand{\semi}{\ltimes}
\newcommand{\orb}{\mathcal O}
\newcommand{\cc}{(\widetilde{U}, G_U, \pi_U)}
\newcommand{\bs}{\backslash}

\DeclareMathOperator{\iso}{Iso}
\DeclareMathOperator{\trace}{Tr}
\DeclareMathOperator{\fix}{Fix}
\DeclareMathOperator{\area}{Area}
\DeclareMathOperator{\vol}{vol}

\begin{document}

Take $\orb_o$ to be some locally orientable orbifold and $\orb_n$ to be
some locally non-orientable orbifold.\\

It is well known that orbifold of different dimensions will have different
Laplace Spectra. So, we only must consider the case in which $\dim(\orb_o)
= \dim(\orb_n)$. Without loss of generality, we will take this dimension to
be odd. 


By Lemma~/*4.5*/ $\orb_o$ will have no even dimensional primary strata while
$\orb_n$ will have at least one even dimensional primary stratum.\\

We first claim that every integer power coefficient in the heat expansion
of $\orb_o$ is $0$. We proceed by contradiction under the assumption that
there exists some nonzero integer power coefficient. So, consider the asymptotic expansion given by

\begin{equation*}
    I_0 + \sum_{N \in S(\orb)} \frac{I_N}{\myabs{\iso(N)}}
\end{equation*}

But, for an odd dimensional orbifold, the $I_0$ will not contribute to
integer power terms. So, we consider only the second part of the sum. We
use /**/ to expand the second half of the sum and arrive at the following
expression.

\begin{equation*}
    \sum_{N \in S(\mathcal{O})}\frac{{(4\pi
    t)}^{-\dim(N)/2}}{\myabs{\iso(N)}}\sum_{k=0}^{\infty}t^k\int_{N}
    \sum_{\gamma \in \iso^{\max}(\widetilde{N})}b_k(\gamma,x) dvol_N
\end{equation*}

In studying the above, we find that in order for $\orb_o$ to have some
nonzero integer power coefficient, $\orb_o$ must have some even dimensional
primary strata. However, this contradicts our earlier conclusion and thus
we reject our assumption and conclude that every integer power coefficient
in the expansion of $\orb_o$ is $0$.\\

We now show that at least one integer power coefficient in the expansion of
$\orb_n$ is nonzero. We have that there exists at least one even
dimensional primary strata in $\orb_n$, so we consider all such strata of
maximal dimension $d$ in $\orb_n$. Note that only these strata of maximal
dimension will contribute to the $-d/2$ term in the heat expansion, which
occurs in the $k=0$ iteration in the sum. Furthermore, by Lemma~/*4.1*/ the
$b_0$ term for each contributing strata is strictly positive. Thus, the
integer $-d/2$ term is the sum of strictly positive terms and is thus
nonzero. This confirms the claim that $\orb_n$ has at least one nonzero
integer power coefficient.\\

So, $\orb_o$ and $\orb_n$ differ in at least one term in the heat
expansion. Specifically, $\orb_n$ will have some nonzero integer power
coefficient in the expansion while the same term in $\orb_o$ is guaranteed
to be $0$.\\

So, any orientable and non-orientable orbifold will have differing heat
expansions, so we conclude that any orientable and non-orientable orbifold
will have different Laplace Spectra. This concludes the proof that we can
hear local orientability.

The proof for the case that $\orb_o$ and $\orb_n$ are of even dimension
proceeds identically to the above with a few small differences. Instead of
considering even dimensional primary stratum, take odd dimensional primary
stratum. And, instead of looking at integer coefficients, take half-integer
coefficient.


\end{document}
