\documentclass[11pt]{article}
\usepackage{amsmath}
\usepackage{amsfonts}

\begin{document}

\title{Rogers 2018 Notes}
%TITLE:
\author{Sean Richardson}
\date{\today}
\maketitle

\section{Week 1}
\subsection{Day 1}
\subsubsection{Talked about:} 
Some goals of research:
\begin{itemize}
    \item Goal 1
    \begin{itemize}
        \item 17 crystolographic groups [TSOT, Thurston's Notes] and associated 2-orbifolds
        \item 14 spherical symmetry groups and associated 2 orbifolds
        \item Generally, see what orbifolds are and relate group to topological object.
    \end{itemize}
    \item Goal 2
    \begin{itemize}
        \item Spectral question for 2-orbifold, what can we hear? (eigenvalue spectrum of Laplace op. on object), (applications)
    \end{itemize}
    \item Goal 3
    \begin{itemize}
        \item Identify a pleasant class of 3-orbifolds to study
        \begin{itemize}
            \item prime 3-orbifold
            \item not prime
            \item Spherical
        \end{itemize}
    \end{itemize}
    \item Goal 4: Repeat Goal 2 but for dim 3.
\end{itemize}
A group action of $G$ on a manifold $M$. This forms a quotient object $M/G$, which can be a manifold or orbifold. (We did example of $G = \mathbb{Z}\times \mathbb{Z}$ and $M = \mathbb{R}^2$). Some questions regarding the quotient of a manifold are: What are the orbits? What is the fundamental domain? What is the gluing diagram?
\subsubsection{TODO}
$M/G$ quotient object if $M = \mathbb{R}^2$ and $G = \{\mathbb{Z}\times\mathbb{Z}, R_{180}\}$
\subsection{Day 2}
\subsubsection{Did}
Quotient object hw
\subsubsection{Talked about}
Homework and first steps into Algebra
\subsubsection{TODO}
\begin{itemize}
    \item Read Dummit-Foote Chapter $1$ on dihedral groups, cyclic groups, definition og a group.
    \item Homework on groups: Looking into $G =\{r_x,r_y,R_{180},\mathbb{Z}\times \mathbb{Z}\}$ and $M = \mathbb{E}$
\end{itemize}
\subsection{Day 3}
\subsubsection{Did}
    hw on groups
\subsubsection{Talked About}
\begin{itemize}
    \item Groups in general. (How many different multiplication table combinations of $e,a,b,c,\dots$ can you have?)
    \item Definition of manifolds and orbifolds. Orbifolds can have nasty complications\dots
\end{itemize}
\subsubsection{TODO}
\begin{itemize}
    \item DGGW example 2.15 
    \item DGGW example 2.16 (3D orbifold)
\end{itemize}
\subsection{Day 4}
\subsubsection{Did}
Both DGGW examples
\subsubsection{Talked about}
The examples and the DGGW paper.
\subsubsection{TODO}
\begin{itemize}
    \item pg. 44 Dummit-Foote, \#4 and \#19.
    \item Gauss-Bonett Theorem in shape of space
    \item Find all isotrop groups for *2222
\end{itemize}
\subsection{Day 5}
\subsubsection{Did}
Gauss-Bonett Theorem and isotrop groups. Perhaps you can find fundamental domain from isotropy groups of points?
\subsubsection{Talked about}
\begin{itemize}
    \item Gauss-Bonett Theorem
    \item Curvature
\end{itemize}
\subsubsection{TODO}
Still pg. 44 Dummit-Foote, \#4 and \#19.
\section{Week 2}
\subsection{Day 6}
\subsubsection{Did}
Dummit-Foote pg. 44 problem 4.
\subsubsection{Talked About}
Addressed some basic set theory, including the formal definition of an equivalence relation to work towards problem 19.
Began our discussion on the heat kernel!
\subsubsection{TODO}
\begin{itemize}
    \item In (;2,2,2,2), Identify the singular strata, their isotropy groups, sort out which of those are ``primary'' singular strata.
    \item Same question for xyz-rotation orbifold.
    \item Perhaps look at Burstall's differential geometry crash course notes
\end{itemize}
\
\subsection{Day 7}
\subsubsection{Did}
Read the important sections of the DGGW paper thoroughly up to section 2.14.
\subsubsection{Talked About}
Worked out the meaning of $Iso^{max}(\tilde{N})$. Talked more about the heat kernel up to the result $Tr(K) = \sum^{\infty}_{j=0}c_jt^{j/2} \sim^{t \rightarrow 0^+} I_0 + \sum_{N \in S(O)}\frac{I_N}{\vert{Iso(N)}\vert}$.
\subsubsection{TODO}
\begin{itemize}
    \item Work out primary strata for some examples
    \item Perhaps look at asymptotic expansion notes
\end{itemize}
\subsection{Day 8}
\subsubsection{Did}
\begin{itemize}
    \item Looked into differential geometry crash course notes, but they are a bit too formal to understand.

\end{itemize}
\end{document}
