\documentclass{article}

\usepackage{amsmath}
\usepackage{amssymb}
\usepackage{amsthm}

\theoremstyle{definition}
\newtheorem{theorem}{Theorem}[section]
\newtheorem{lemma}[theorem]{Lemma}
\newtheorem{definition}[theorem]{Defintion}

\begin{document}

\title{Odd Dimensional Orbifolds}
\author{Liz Stanhope and Sean Richardson}
\date{\today}
\maketitle

\section{}

\begin{definition}
    A chart $(\tilde{U},G_U,\pi_U)$ on some orbifold is \emph{orientable} if the group $G_U$ consists of orientation-preserving transformations of the open set $\tilde{U} \subset \mathbb{R}^n$. An orbifold $\mathcal{O}$ is \emph{locally orientable} if every chart on $\mathcal{O}$ is orientable. Conversely, an orbifold $\mathcal{O}$ is \emph{locally non-orientable} if there exists a single chart on $\mathcal{O}$ that is not orientable.
\end{definition}

\begin{lemma}
    Let $\mathcal{O}$ be a locally orientable orbifold. If $\dim(\mathcal{O})$ is odd, then there exists no primary singular stratum $N$ such that $\dim(N)$ is even. Similarly, if $\dim(\mathcal{O})$ is even, then there exists no primary singular stratum $N$ such that $\dim(N)$ is odd.
\end{lemma}

\begin{theorem}
    Consider a locally orientable orbifold $\mathcal{A}$ and a locally non-orientable orbifold $\mathcal{B}$. If $\dim(\mathcal{A}) = \dim(\mathcal{B})$ is odd, then there exists no $\mathcal{A}$ and $\mathcal{B}$ that are isospectral.
\end{theorem}

\end{document}
