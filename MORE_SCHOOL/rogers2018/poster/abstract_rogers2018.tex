\documentclass{article}

\usepackage{amsmath}
\usepackage{amssymb}

\newcommand{\myabs}[1]{\vert#1\vert}
\newcommand{\pder}[2]{\frac{\partial#1}{\partial#2}}

\DeclareMathOperator{\iso}{Iso}
\DeclareMathOperator{\fix}{Fix}
\DeclareMathOperator{\tr}{Tr}

\begin{document}

\date{\today}
\title{Abstract Draft}
\author{Sean Richardson and Liz Stanhop}
\maketitle

\section{Abstract}

\hspace{\parindent} Given some object, such as a drum, there exist a spectrum
of fundamental frequencies determined by physics. However, if the drum was in a
neighboring room and you could only listen to these frequencies, is it possible
to deduce the drum's shape? In other words, ``Can you hear the shape of a
drum?''. In this research project, we ask a similar question but for abstract
mathematical objects. The vibrational frequencies for abstract objects are
defined with the eigenvalue solutions of the Laplace operator on the wave form.
The list of solutions $\lambda_1, \lambda_2, \dots$ is called the Laplace
Spectra where $\sqrt{\lambda_i}$ is a valid fundamental frequency. 

The abstract shapes we study are called \emph{orbifolds}. An orbifold a normal multidimensional surface that is allowed to have some ``trouble spots'', which are tied to the symmetries allowed in $n$ dimensional space. Formally, an orbifold is a generalization of a manifold such that it can have local structure of $\mathbb{R}^n/G$ where $G$ is a group of isometries.

I now present our research question: given the Laplace Spectra of some unknown orbifold $\mathcal{O}$, what properties of $\mathcal{O}$ are determined?

Our result requires the new definition of \emph{local orientability}. An orbifold $\mathcal{O}$ is \emph{locally non-orientable} if there exists a short Orientation-reversing path on $\mathcal{O}$. If no such path exists, $\mathcal{O}$ is \emph{locally orientable}.

We found that it is possible to hear the locally orientability of an orbifold. Formally, there exists no locally orientable orbifold and locally non-orientable orbifold with the same Laplace Spectra.

\section{Notes}
\begin{itemize}
    \item Where can I cut down?
    \item Is ``wave form'' the right way to describe $\psi(\mathbf{x})$? [from $u(t,\mathbf{x}) = A(t)\psi({\mathbf{x}})$]
\end{itemize}

\end{document}

