\documentclass{article}

\usepackage{amsmath}
\usepackage{amssymb}

\newcommand{\myabs}[1]{\vert#1\vert}
\newcommand{\pder}[2]{\frac{\partial#1}{\partial#2}}

\DeclareMathOperator{\iso}{Iso}
\DeclareMathOperator{\fix}{Fix}
\DeclareMathOperator{\tr}{Tr}
\DeclareMathOperator{\vol}{Vol}

\setlength\parindent{0pt}

\begin{document}

\section{Abstract}


\hspace{\parindent} Given some object, such as a drum, there exists a
spectrum of fundamental frequencies determined by physics. However, if the
drum was in a neighboring room and you could only listen to these
frequencies, is it possible to deduce the drum's shape? In other words,
``Can you hear the shape of a drum?''. In this research project, we ask a
similar question but for abstract mathematical objects. The vibrational
frequencies for abstract objects correspond to the eigenvalue spectrum of
the Laplace operator associated to the object. 

The abstract shapes we study are called \emph{orbifolds}. An orbifold is a
multidimensional object that is allowed to have some ``trouble spots,''
which are tied to the symmetries allowed in $n$-dimensional space. We ask:
Given the Laplace spectrum of an unknown orbifold $\mathcal{O}$, what
properties of $\mathcal{O}$ are determined?  We show that one can hear the
\emph{local orientability} of an orbifold.  That is, we can use the Laplace
spectrum to detect trouble spots associated to orientation reversing
symmetries of $n$-dimensional space.


\section{Symmetries in Space}

2 examples..
symmetrical b/c can perform some action on the piece of paper that leaves
the paper unchanged.
\\- $\Gamma$.
\\- isometries
\\- orientation preserving
\\- orientation reversing.

Introduce strata?

\section{Orbifold}
\subsection{Intuitive}

We consider abstract objects called \emph{orbifolds}. Visually, we
represent $2$ dimensional orbifolds as a surface in space with a few
``trouble spots''. 
\\- 2 dimensions: torus with ``trouble spots''
\\- In this case, the trouble spots correspond to symmetries of $2$
dimensional space
\\- rotational symmetry
\\- reflectional symmetry

Gets ``folded'' in half or ``twisted'' like a party hat, resulting in 



\subsection{Formal}

Formally, an \emph{orbifold} is a generalization of the Riemannian
Manifold (some $n$ dimensional surface). While a manifold requires local
structure of $\mathbb{R}^n$, an orbifold allows local structure of
$\mathbb{R}^n/\Gamma$ where $\Gamma$ is a group of isometries.

\section{Laplace}
\subsection{Intuitive/motivation}
When you pluck a string, the string produces a sound determined by a
specific resonance frequency it can vibrate at. Actually, the string can
vibrate at a \emph{spectrum} of resonance frequencies. Furthermore, all
objects, not just strings, vibrate at a certain spectrum as observed by the
sounds that drums produce. Mathematically, we express these frequencies
through the \emph{Laplace Spectra}.

\subsection{Formal}
The Laplace Spectra follows from the solutions to the following PDE called
the ``wave equation''.
$$ \Delta u(t,\mathbf{x}) = \pder{u(t,\mathbf{x})}{t} $$
\\
(formal definition of the Laplace Spectra)

I will now give a more mathematically grounded definition of the Laplace
Spectra. Consider some manifold $\mathcal{M}$. Let $u(t,\mathbf{x})$ be the
displacement of some point $x$ on $\mathcal{M}$ at time $t$ from
equilibrium. (picture). We then define the energy of the manifold
$E_{\mathcal{M}}$ to behigher the faster the surface moves and the more
stretched it is. /*get exact formula?*/ In applying conservation of energy
($\pder{E_{\mathcal{M}}}{t} =0$), we derive the following PDE known as the
wave equation

\begin{equation}\label{eq:test}
    \Delta u(t,\mathbf{x}) = \frac{d^2 u}{dt^2}
\end{equation}
We are looking for fundamental frequencies, which we define to be waves of the form $u(t,\mathbf{x}) = A(t)\psi(\mathbf{x})$. With this, we break down the wave equation into the following.
\begin{equation}
    \Delta \psi(\mathbf{x}) = -\lambda \psi(\mathbf{x})
    \text{ and }
    \Delta A(t) = -\lambda A(t)
\end{equation}
For some $\lambda$. Only discrete values of $\lambda$ solve this equation. These values are represented $\lambda_1, \lambda_2, \dots$ and called the Laplace Spectra.

\section{The Question:}
Given a specific drum, it is possible to deduce what sound the drum will make when hit. However, consider the reverse: if you hear a drum in the neighboring room, is it possible to deduce the drum. Put nicely by /**/, can you hear the shape of a drum? This is what the field of \emph{Inverse Spectral Geometry} attempts to answer. In this research, we apply Inverse Spectral Geometry techniques to Orbifolds.

Formally, consider some laplace spectra $\lambda_1, \lambda_2, \dots$ (as defined in /**/) belonging to some unknown orbifold $\mathcal{O}$ (as defined in /**/). From the laplace spectra, what properties can we deduce about $\mathcal{O}$?

\section{Asymptotic Heat Expansion Technique}
\subsection{Intuitive}
Consider heating a point $\mathbf{x}$ on some orbifold $\mathcal{O}$ with a
match. Then, allow the heat to disperse around $\mathcal{O}$. The
temperature of the initial value will decrease.

\subsection{Formal}
The solution to the following PDE (the heat equation)
$$-\Delta u(t,\mathbf{x}) =  \pder{u(t,\mathbf{x})}{t}$$
is of the form $u(t,\mathbf{x}) = \int_{\mathcal{M}}
K(t,\mathbf{x},\mathbf{y}) \mu_0dvol
\mathcal{M}_y$. \\
$$\tr(K) = \sum_{j=1}^{\infty}e^{-\lambda_j t} \sim a_0 t^{1}$$

\section{Results}

\subsection{Locally Orientable}
First, we define \emph{locally orientability}. 
\subsubsection{Intuitive}
Intuitively, \dots

\subsubsection{Formal}

If for $g \in \Gamma$, $\det(g) = -1$. $\Gamma$ is orientation reversing.
A chart $(.,.,.)$ on $\mathcal{O}$ is defined to be $\emph{orientable}$ if there exists
some orientation reversing operation in $\Gamma$.
\end{document}
