\documentclass[12pt]{amsart}
\usepackage{multicol}
\usepackage[margin=0.5in]{geometry}
\setlength{\parindent}{0cm}

\usepackage{siunitx}


\DeclareMathOperator{\proj}{\mathbf{proj}} 
\DeclareMathOperator{\lcm}{lcm}



\newcommand{\deriv}[2]{\frac{d#1}{d#2}}
\newcommand{\myabs}[1]{\vert#1\vert} 


\begin{document}
\title{GRE Prep}
\author{Sean Richardson}
\maketitle

\begin{multicols}{2}
\section{Precalculus}
\textbf{Bits and Bobs:}
\begin{itemize}
    \item \emph{Parabola}: Let $F$ be a point and $D$ a fixed line that
        doesn't contain $F$. A parabola is the set of points in the plane
        equidistant from $F$ and $D$. Further, for $y = \frac{1}{4p} x^2$, $F = (0,p)$ and $D$ is given by $y = -p$.
    \item \emph{Hyperbola}: The set of all points in plane such that the
    difference between the distances of two fixed points (the \emph{foci})
    is constant. Further, for $\frac{x^2}{a^2} - \frac{y^2}{b^2} = 1$, the
    foci are given by $(\pm c, 0)$ where $c = \sqrt{a^2+b^2}$. (Can find
        asyptotes and vertices).
    \item \emph{Elipse}: Thse set of all points in the plane 
    \item Fundamental theorem of algebra.
    \item Rational roots Thm [18] To remember: $0 = a_1x+a_0$ /*pf*/
    \item Conjugate radical roots theorem [18] /*remember with quadratic
        formula*/. /*pf*/
    \item Complex conjugate roots thm
    \item Sum of roots of polynomial is $-\frac{a_{n-1}}{a_n}$. /*pf*/
    \item Product of roots of polynomial is $(-1)^n\frac{a_0}{a_n}$. /*pf*/
\end{itemize}
\textbf{Trig:}
\begin{itemize}
    \item $1+\tan^2(\theta) = \sec^2(\theta)$.
    \item $1+\cot^2(\theta) = \csc^2(\theta)$.
    \item $\sin(\alpha+\beta) =
        \sin(\alpha)\cos(\beta)+\sin(\beta)\cos(\alpha)$.
    \item $\cos(\alpha+\beta) = \cos(\alpha)\cos(\beta) -
        \sin(\alpha)\sin(\beta)$.
    \item $\tan(\alpha+\beta) =
     \frac{\tan(\alpha)+\tan(\beta)}{1-\tan(\alpha)\tan(\beta)}$.
    \item $\sin(2\theta) = 2\sin(\theta)\cos(\theta)$.
    \item $\cos(2\theta) = \cos^2(\theta)-\sin^2(\theta) =
        1-2\sin^2(\theta) = 2\cos^2(\theta)-1$.
    \item $\tan(2\theta) = \frac{2\tan(\theta)}{1-\tan^2\theta}$
    \item $\sin(\frac{\theta}{2}) = \pm \sqrt{\frac{1-\cos\theta}{2}}$.
    \item $\cos(\frac{\theta}{2}) = \pm \sqrt{\frac{1+\cos\theta}{2}}$.
    \item $\tan(\frac{\theta}{2}) = \frac{\sin\theta}{1+\cos\theta}$.
\end{itemize}
\section{Calculus I and II}%
\begin{itemize}
    \item sequence convergence rules
    \item limit convergence rules
    \item Squeeze theorem
    \item L'Hopital's rule
    \item Extreme \& Intermediate value theorems
    \item Definition of derivative
    \item Derivative of inverse function:
        $(f^{-1})^{'}(y_0)=\frac{1}{f^{'}(x_0)}$ if $y_0 = f(x_0)$
    /*how to remember*/
    \item Implicit differentiation
    \item Mean value theorem
    \item Integration by parts
    \item Fundamental Theorem of Calculus (both forms).
    \item Solids of revolution
    \item Arc length of curve $y(x)$ given by 
        \begin{equation}
            s = \int_{x_i}^{x_f} \sqrt{{\left(\deriv{y}{x}\right)}^2+1}\ \ dx
        \end{equation}
    \item derivative of $f(x)^{g(x)}$ things.
    \item $1+x+\dots+x^n = \frac{1-x^{n+1}}{1-x}$. And can take $n \to \infty$ for $\myabs{x}$
    \item p-series
    \item Comparison test
    \item Ratio test
    \item Integral test
    \item Root test \dots $\lim_{n\to\infty}(a_n)^{ \frac{1}{n}}$.
    \item Interval of convergence of power series: use ratio test and solve for $x$. (endpoints must be checked case by case).
    \item Taylor Series given by
        \begin{equation}
        f(x) = \sum_{n=0}^\infty \frac{f^{(n)}(a)}{n!}{(x-a)}^n 
        \end{equation}
    \item Taylor's theorem / Taylor Series error thing.
\end{itemize}
\textbf{Derivatives to know:}
\begin{itemize}
    \item $\deriv{}{x}(a^x) = \log(a)a^x$.
    \item $\deriv{}{x}(\log_a(x)) = \frac{1}{x\log(a)}$.
    \item $\deriv{}{x}(\tan x) = \sec^2(x)$.
    \item $\deriv{}{x}(\cot x) = -\csc^2(x)$.
    \item $\deriv{}{x}(\sec x) = \sec x \tan x$.
    \item $\deriv{}{x}(\csc x) = -\csc x \cot x$.
    \item $\deriv{}{x}(\arcsin x) = \frac{1}{\sqrt{1+x^2}}$.
    \item $\deriv{}{x}(\arccos x) = \frac{-1}{\sqrt{1+x^2}}$.
    \item $\deriv{}{x}(\arctan x) = \frac{1}{1+x^2}$.
\end{itemize}
\textbf{Integrals to know:}
\begin{itemize}
    \item $\int a^x dx = \frac{1}{\log a} a^x + c$.
    \item $\int \sec^2 x dx = \tan x + c$.
    \item $\int \csc^2 x dx = -\cot x + c$.
    \item $\int \sec x \tan x dx = \sec x + c$.
    \item $\int \csc x \cot x dx = -\csc x + c$.
    \item $\int \frac{1}{\sqrt{1-x^2}} dx = \arcsin x + c$.
    \item $\int \frac{1}{1+x^2} dx = \arctan x + c$.
\end{itemize}
\textbf{Taylor Series to know:}
\begin{itemize}
    \item $ \frac{1}{1-x} = \sum^{\infty}_{n=0} x^n$, $-1 < x < 1$
    \item $ \frac{1}{1+x} = \sum^{\infty}_{n=0} {(-1)}^n x^n$, $-1 < x < 1$
    \item $ \log(1-x) = -\sum^{\infty}_{n=0} \frac{x^n}{n}$, $-1 \leq x < 1$
    \item $ e^x = \sum^{\infty}_{n=0} \frac{x^n}{n!} $, all $x$.
    \item $ \sin x = \sum^{\infty}_{n=0} \frac{{(-1)}^n}{(2n+1)!}x^{2n+1}$, all $x$
    \item $ \cos x = \sum^{\infty}_{n=0} \frac{{-1}^n}{(2n)!}x^{2n}$, all $x$
\end{itemize}
\textbf{Trig Substitution Method:}\\
For integrals that contain $\sqrt{a^2-x^2}$, $\sqrt{a^2+x^2}$, or $\sqrt{u^2-x^2}$. In particular:
\begin{center}
    \begin{tabular}{c c}
        \textit{If integrand contains} & \textit{Make this substitution} \\
        $\sqrt{a^2-u^2}$               & $x = a\sin\theta$\\
        $\sqrt{a^2+u^2}$               & $x = a\tan\theta$\\
        $\sqrt{u^2-a^2}$               & $x = a\sec\theta$\\
    \end{tabular}
\end{center}
\textbf{Partial Fractions Method:}\\
For integrals of the form $\int \frac{P(x)}{Q(x)} dx$, with $P(x), Q(x)$ polynomials, $\deg(P) < \deg(Q)$. The method:
\begin{enumerate}
    \item First, factor $Q(x)$.
    \item Express $\frac{P(x)}{Q(x)} = \frac{A_1}{q_1(x)} + \frac{A_2}{q_2(x)} \dots$. The $q_i$'s are factors of $Q$. If $Q$ factors with a term of th form $(ax+b)^n$, there will be $n$ corresponding partial fractions $(ax+b), (ax+b)^2, \dots$. If you get irreducible quadratic, have a degree $1$ numerator.
    \item Solve for $A_i$'s by multiplying both sides by corresponding $q_i$ and plugging in root.
\end{enumerate}
\section{Multivariable Calculus}%
\label{sec:multivariable_calculus}

\begin{itemize}
    \item Projection of $ \mathbf{b}$ onto $ \mathbf{a}$:
    \begin{equation*}
            \proj_{\mathbf{a}}\mathbf{b} = \frac{ \mathbf{a} \cdot \mathbf{b}} { \mathbf{a} \cdot \mathbf{a}} \mathbf{a}.
    \end{equation*}
    \item Triple product is volume of paraleltope
    \item Various vector product identities?
    \item Line equations in space
    \item Plane equations in space
    \item Various coordinate systems
    \item Tangent plane to surface and linear approxmations
    \item Higher order approximations /*TODO*/
    \item Chain rule
    \item Gradient and properties
    \item Min / max problems /*2nd deriv test TODO*/
    \item Min / max problems with constraint: if possible combine constraint and function into simple function and apply calc 1.
    \item Otherwise \dots Lagrange multiplier method. Function $f$ and constraint $g = c$. Then, impose $\nabla f = \lambda \nabla g$.
    \item Line integrals and arclength, integrating on V.F etc.
    \item Fundamental theorem of calc for line integrals.
    \item Green's theorem and applications.
    \item Weird cases /*[158--159]*/
\end{itemize}
\section{Differential Equations}%
\label{sec:differential_equations}

\section{Number Theory}%
\label{sec:number_theory}

\begin{itemize}
    \item Relating divisibility to digits rules
    \item Division algorithm
    \item $\gcd$ and $\lcm$ definitions and relation to prime factorizations (min and max).
    \item Also, $\gcd(a,b) \cdot \lcm(a,b) = ab$.
    \item Euclidean algorithm.
    \item Fermat's Little Theorem. If $p$ prime, $p \not| a$, then
        \begin{equation*}
            a^{p-1}\equiv1(\text{mod }p)
        \end{equation*}
\end{itemize}

    \textbf{Diophantine Equation} $ax+by=c$. Has solution iff $\gcd(a,b) | c$. Given solution $(x_1,y_1)$ all solutions are given by:
    \begin{equation*}
        x = x_1+ t\frac{b}{\gcd(a,b)} \text{ and } y = y_1- t\frac{a}{\gcd(a,b)} 
        \end{equation*}
        for $t \in \mathbb{Z}$.
\end{multicols}
\end{document}
