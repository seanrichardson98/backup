\documentclass[12pt]{amsart}

\setlength{\parindent}{0cm}
\usepackage[margin=1in]{geometry}

\usepackage{amsmath}
\usepackage{amsthm}

\usepackage{float}
\usepackage{enumitem}
\usepackage{commath}
\usepackage{tikz}
\usetikzlibrary{calc}
\usepackage{cancel}
\usepackage{subfigure}
\usepackage{multicol}
\usepackage{gensymb}

\usepackage{graphicx}
\graphicspath{{graphics/}}

\usepackage{my_notes}
\usepackage{my_math}

\theoremstyle{definition}
\newtheorem{theorem}{Theorem}
\newtheorem{definition}[theorem]{Definition}
\newtheorem{corollary}[theorem]{Corollary}
\newtheorem{technique}[theorem]{Technique}

\newenvironment{just}{\textit{Justification.}}{}
\newenvironment{trans}{\textit{Translation.}}{}

\newenvironment{mybox}
{\begin{tcolorbox}[colback=red!5!white,colframe=red!75!black]}
{\end{tcolorbox}}

\newenvironment{mytbox}[1]
{\begin{tcolorbox}[colback=red!5!white,colframe=red!75!black,title=#1]}
{\end{tcolorbox}}

\def\mathunderline#1#2{\color{#1}\underline{{\color{black}#2}}\color{black}}

\newcommand{\cH}{\mathcal{H}}
\newcommand{\cE}{\mathcal{E}}

\begin{document}
\title{Combinatorics 2}
\author{Sean Richardson}
\date{\today}
\maketitle

\begin{definition}[Hypergraph]
    A \emph{hypergraph} $\cH$ is given by $\cH = (V,\cE)$ for a set
    \emph{vertex set} $V$ and \emph{edge set} $\cE$: a collection of subsets
    of $V$.  Additionally: $V,\cE \neq \O$. Note we allow repeated edges
    and $\cE \ni e = \O$.
\end{definition}
\begin{definition}[Simple Hypergraph]
    A hypergraph $\cH$ with no \emph{multiedges} (there are no repeated
    edges in $\cE$).
\end{definition}
\begin{definition}[Singleton, Isolated]
    An edge $e \in \cE$ is a \emph{singleton} if $\abs{e} = 1$. A vertex $v
    \in V$ is \emph{isolated} if no edge contains it.
\end{definition}
\begin{definition}[Degree]
    
\end{definition}

\end{document}
