\documentclass[12pt]{amsart}
\usepackage{multicol}
\usepackage[margin=0.5in]{geometry}
\setlength{\parindent}{0cm}

\begin{document}
\title{Putnam Summary}
\author{Sean Richardson}
\maketitle
\begin{multicols}{2}
\section{General}
\subsection{Working Flowchart}
\begin{enumerate}
    \item Understand the problem. (Play around with the problem in whatever
        way I find intuitive / test something concrete)
    \item Select good notation to symbolically represent the problem.
    \item Write down known information symbolically and draw a diagram if applicable.
    \item Write down what to show symbolically. Brainstorm what would be
        sufficient to prove.
    \item Brainstorm all relevant approaches to the problem (see approaches
        listed below).
    \item Identify relevant areas of math and brainstorm all relevant tools
        for the problem (see tools listed below).
    \item Choose what seems to be the best approach.
    \item Play around with tools on the problem with this approach
        \emph{algebraically} (start
        with a specific/easier instance of the problem if possible). Make
        sure to incorporate all necessary information.
    \item If stuck, brainstorm possible instances of ``if stuck'' ideas.
    \item If appears to be a dead end, write down possible new
        tools/approaches, consider a new approach and repeat.
\end{enumerate}

\subsection{If Stuck}
\begin{itemize}
    \item Consider special case / simplified version of problem. What would
        be the easiest thing to prove?
    \item Brainstorm intermediate steps to solving the problem. What would
        be sufficient to prove?
    \item Derive consequence of problem and try to show that first.
    \item Reformulate problem. (contrapositive, contradiction, substitution, clever manipulation).
\end{itemize}

\subsection{General Approaches}
\begin{itemize}
    \item Contradiction.
    \item Induction
    \item Construction
    \item Invariant
    \item Clever Reformulation
    \item Clever Choice
    \item Weaker Claim
    \item Stronger Claim
\end{itemize}

\subsection{General Tools}
\begin{itemize}
    \item Pigeonhole Principle
\end{itemize}
\section{Number Theory}
\subsection{Approaches}
\begin{itemize}
    \item Manipulate problem so we are dealing with integers on both sides.
    \item Modular arithmetic / even vs odd argument.
\end{itemize}
\subsection{Tools}
\begin{itemize}
    \item Fundamental theorem of arithmetic.
    \item $\gcd$ theorem.
    \item Chinese Remainder Theorem.
    \item Fermat-Euler Theorem / Fermat's Little Theorem.
    \item Division algorithm / euclidean algorithm for $\gcd$.
    \item Euclid's Lemma
    \item Wilson's Theorem.
    \item Linear Diophantine Equation theory.
    \item Pell's Equation theory.
\end{itemize}

\section{Algebra}
\subsection{Approaches}
\begin{itemize}
    \item Algebraic manipulation of problem into simpler form.
    \item Imposing symmetry on the problem to make things (ex: factorization) more obvious.
\end{itemize}
\subsection{Tools}
\begin{itemize}
    \item Geometric sum formula (finite and infinite). Useful for factoring things, evaluating telescoping products and sums.
    \item $x^2 > 0$
    \item Division algorithm for polynomials
    \item Fundamental Theorem of Algebra.
    \item Roots and Divisibility relationship
    \item A polynomial can only have a finite number of roots (if can establish infinite pattern, use this!)
    \item Rational Roots Theorem
    \item Vieta's Relations: Given\\ $P(x) = a_n x^n+a_{n-1}x^{n-1}+\cdots+a_1x+a_0 = a_n(x-\alpha_1)(x-\alpha_2)\cdots(x-\alpha_n)$, then \dots /**/
    \item Unique Factorization of Polynomials (even for multivariabled polynomials).
    \item $x^n-y^n = $
    \item $x^n+y^n = $
    \item $x^3 + y^3 + z^3 - 3xyz = $
    \item $(x+y)^p$ for primes $p$.
    \item Lagrange Identity
    \item Cauchy-Schwartz Inequality
    \item Bernoulli's Identity
    \item Cauchy's Inequality
    \item Triangle Inequality
    \item Reverse Triangle Inequality
    \item Euler's identities on products of sums of squares.
    \item Sophie Germaine Identity
    \item (other algebraic manipulations)
\end{itemize}
\section{Geometry}
\subsection{Approaches}
\subsection{Tools}
\begin{itemize}
    \item Cross, dots products
    \item $\vec{a} \times (\vec{b} \times \vec{c}) = (\vec{c} \times \vec{a}) \vec{b} - (\vec{b} \times \vec{a}) \vec{c}$
    \item $\vec{u} \cdot (\vec{v} \times \vec{w}) = \vec{w} \cdot (\vec{u} \times \vec{v})$
    \item 
\end{itemize}
\section{Linear Algebra}
\subsection{Approaches}
\subsection{Tools}
\begin{itemize}
    \item Cauchy-Schwartz Inequality
    \item $m$ homogeneous linear equations with $n$ variables. $m < n$ implies
        there exists a nontrivial solution.
    \item Trace
    \item Eigenstuff
    \item Sum of eigenvalues is the trace
\end{itemize}

\section{Combinatorics and Graph Theory}
\subsection{Approaches}
\begin{itemize}
    \item Count something two different ways
    \item Vector space dimension bound
    \item Homogeneous System of Linear Equations Method
    \item Probability Method
\end{itemize}

\subsection{Tools}
\begin{itemize}
    \item Pigeonhole Principle
    \item Inclusion-Exclusion Principle
    \item Binomial Theorem / Multinomial Theorem
\end{itemize}

\section{Analysis, Calculus, and Diff eq}
\subsection{Approaches}
\begin{itemize}
    \item Prove for $\mathbb{Z}$, then $\mathbb{Q}$, then $\mathbb{R}$.
    \item Apply definite integrals to differential equations
    \item Associate given differential equation with (Euler method inspired
        sequence) and vice versa
\end{itemize}
\subsection{Tools}
\begin{itemize}
    \item Fundamental Theorem of Calculus in various forms
    \item Mean Value Theorem
    \item Intermediate Value Theorem
    \item Definition of derivative
    \item Taylor's Theorem / Taylor Series
    \item $f(x)\leq g(x)$ on $a \leq x \leq b$, then we have $\int_a^b f(x) dx \leq \int_a^b g(x) dx$.
    \item Integration by substitution: Basic techniques, trig substitution techniques, technique on simple fractions.
    \item Integration by parts techniques.
    \item Integration by partial fractions.
    \item Can take advantage of complex numbers to simplify integral.
    \item Continuity (multiple definitions)
    \item Every bounded monotonic sequence is convergent
    \item For every real number $x$, there is a rational sequence
        converging to $x$.
    \item Integral test. Specifically, if $g$ is a positive decreasing function, $\int_{1}^{n+1}g(x)dx \leq g(1) + \cdots + g(n) \leq \int_{0}^{n}g(x)dx$. (In particular, $\ln$).
    \item Separable differential equations solution
    \item Functions that grow at a more than proportional rate to function
        itself blow up in finite time.
    \item Additive functions
    \item Hamel Basis
\end{itemize}
\section{Set Theory and Abstract Algebra}
\subsection{Approaches}
\begin{itemize}
    \item Impose an order on your set.
\end{itemize}
\subsection{Tools}
\begin{itemize}
    \item Order, total order.
\end{itemize}
\end{multicols}
\end{document}
