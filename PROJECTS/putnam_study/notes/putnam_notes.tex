\documentclass[12pt]{amsart}

\setlength{\parindent}{0cm}
\usepackage[margin=0.8in]{geometry}

\usepackage{amsmath}
\usepackage{amsthm}
\usepackage{mathtools}

\usepackage{float}
\usepackage{enumitem}
\usepackage{commath}
\usepackage{tikz}
\usetikzlibrary{calc}
\usepackage{cancel}
\usepackage{subfigure}
\usepackage{multicol}
\usepackage{gensymb}
\usepackage{enumitem}

\usepackage{graphicx}
\graphicspath{{graphics/}}

\usepackage{my_notes}
\usepackage{my_math}

\theoremstyle{definition}
\newtheorem{theorem}{Theorem}[section]
\newtheorem{definition}[theorem]{Definition}
\newtheorem{lemma}[theorem]{Lemma}
\newtheorem{tool}[theorem]{Tool}
\newtheorem{approach}[theorem]{Approach}
\newtheorem{corollary}[theorem]{Corollary}

\newenvironment{just}{\textit{Justification.}}{}
\newenvironment{trans}{\textit{Translation.}}{}

\newenvironment{mybox}
{\begin{tcolorbox}[colback=red!5!white,colframe=red!75!black]}
{\end{tcolorbox}}

\newenvironment{mytbox}[1]
{\begin{tcolorbox}[colback=red!5!white,colframe=red!75!black,title=#1]}
{\end{tcolorbox}}

\def\mathunderline#1#2{\color{#1}\underline{{\color{black}#2}}\color{black}}

\newcommand{\Mod}[1]{\ (\mathrm{mod}\ #1)}

\begin{document}
\title{Putnam}
\author{Sean Richardson}
\maketitle

\section{General Advice}

Before you even begin to attempt the problem, there are a few important
steps:
\begin{enumerate}
    \item Understand the Problem/Question.
    \item Select good notation to symbolically represent problem
    \item Write down all known information and what needs to be shown in the
        selected notation. Draw a diagram if applicable.
\end{enumerate}

Would recommend writing down all known information on one page for
something to stare at. As new information is gained, add it to the page.
Once this is completed, you can continue as follows:

\begin{enumerate}[resume]
    \item Brainstorm all relevant approaches to the problem (see approaches
        listed below).
    \item Identify relevant areas of math and brainstorm all relevant tools
    for the problem (see tools listed below).
    \item Choose what seems to be the best approach.
    \item Play around with tools on the problem with this approach
    \emph{algebraically} (start
    with a specific/easier instance of the problem if possible). Make
    sure to incorporate all necessary information.
    \item If I get stuck, consider a new approach and repeat.
    \item If run out of approaches, brainstorm ``if stuck'' ideas.    
\end{enumerate}

If stuck or you need new information, can try the following:
\begin{itemize}
    \item Consider special case / simplified / generalized version of problem
    \item Formulate conjecture that would imply problem and try to prove
        conjecture
    \item Derive consequence of problem and try to show that first.
    \item Reformulate problem. (contrapositive, contradiction,
        substitution, clever manipulation).
\end{itemize}

Now on to all of the various approaches and tools to attempt to apply:
\subsection{General Approaches}
\begin{approach}[Proof by Contradiction]
    If supposing the opposite of the claim gives some concrete information
    to manipulate, supposing the opposite could be the best way to start.
\end{approach}
\begin{approach}[Proof by Induction] If trying to prove for a general $n$,
    induction is likely a good place to start. In trying to get the
    inductive step, you could try working with a concrete small $n$ and try
    to prove it using smaller terms. Remember, can also apply strong
    induction, suppose a stronger inductive hypothesis, induct in reverse,
    induct for $2^n$ first, or other variations on the idea.
\end{approach}
\begin{approach}[Weaker Claim]
    Is it sufficient to prove a weaker version of claim or only prove the
    claim for some subset of cases? \textit{Such as proving a claim only
    for primes implies $\mathbb{N}$ in some cases.}
\end{approach}
\begin{approach}[Stronger Claim]
    Is it easier to prove a stronger version of the claim?  \textit{This is
    particularly useful in induction proofs, for it gives a stronger
    induction hypothesis. And in many other situations a stronger claim
    gives more tools.}
\end{approach}
\begin{approach}[Reframe Problem]
    Is it possible to move the problem to a different and easier setting by
    an isomorphism of some sort? \textit{To find such a setting, first
    identify the relevant aspects of the question, and find settings with
    the same relevant aspects}
\end{approach}
\begin{approach}[Clever Choice]
    Do you have control over some variable / freedom to choose a specific
    value or some subset of values? Perhaps a clever choice of what you
    have control over will make things fall into place.
\end{approach}
\begin{approach}[Invariant]
    Useful for proving nonexistence. In showing that some
    process/construction is impossible, find an \emph{invariant} --- an
    unchanged value --- that cannot take on a sufficient value for the
    existence.
\end{approach}
\subsection{General Tools}
\begin{tool}[Pigeonhole Principle] If trying to show existence, this is a
    common technique that can come up in combinatorics, number theory, and
    a variety of unexpected places.
\end{tool}
\section{Number Theory}
\subsection{Approaches}
\begin{approach}[Algebraic Maniuplation into Integers] If possible, get the
    equation / statement in the form of integers. This will allow the
    application of all the number theory designed for integers below.
\end{approach}
\begin{approach}[Modular Arithmetic] A common approach to getting problems
    into a simpler form. 
\end{approach}
\subsection{Tools}
\begin{tool}[Fundamental Theorem of Arithmetic]\end{tool}
\begin{tool}[$\gcd$ theorem]\end{tool}
\begin{tool} If $x \in \mathbb{Z}, x \neq 0$ then $x \geq 1$. \end{tool}
\begin{tool}[Chinese Remainder Theorem]\end{tool}
\begin{tool}[Fermat-Euler Theorem]
    Recall the Euler function $\phi(n)$ gives the number smaller terms that
    are coprime to $n$. Now to theorem,\\
    If $a$ and $m$ are coprime, then
    \begin{equation*}
        a^{\phi(m)} \equiv 1\Mod m
    \end{equation*}
\end{tool}
\begin{tool}[Fermat's Little Theorem]
    A corollary of Euler-Fermat, given $p$ prime and $a$ integer,
    \begin{equation*}
        a^p \equiv a \Mod p
    \end{equation*}
\end{tool} 
\begin{tool}[Division Algorithm]\end{tool}
\begin{tool}[Euclidean Algorithm]\end{tool}
\begin{tool}[Euclid's Lemma] This is the common definition of prime\end{tool}
\begin{tool}[Wilson's Theorem]
    If $p$ is prime then $(p-1)! \equiv -1 \Mod p$ 
\end{tool}
\begin{tool}[Linear Diophantine Equations]/*TODO*/\end{tool}
\begin{tool}[Pell]\end{tool}
%ALGEBRA
\section{Algebra}
\subsection{Approaches}
\begin{approach}[Maniuplation]
    Always be on the lookout for an algebraic manipulation of the problem
    into a simpler form.
\end{approach}
\subsection{Tools}
\begin{tool}[Geometric Sum Formula]
    \begin{equation*}
        \sum_{k=0}^nx^k = \frac{1-x^{n+1}}{1-x}
    \end{equation*}
    How to remember: Call the sum $\mathcal{S}$, to make things cancel, do
    $\mathcal{S}-x\mathcal{S} = 1-x^{n+1}$
\end{tool}
\begin{tool} $x^2 \geq 0$. More useful than you'd think. \end{tool}
\begin{tool}[Lagrange Identity]
    \begin{equation*}
        \sum_{k=1}^n a_k^2 \sum_{k=1}^n b_k^2
        - \left(\sum_{k=1}^n a_k b_k \right)^2
        = \sum_{i<k}(a_ib_k-a_kb_i)^2
    \end{equation*}
    How to remember: /*TODO*/
\end{tool}
\begin{tool}{Cauchy-Schwarz Inequality}
    Following from Lagrange Identity and $x^2 > 0$:
    \begin{equation*}
        \sum_{k=1}^n a_k^2 \sum_{k=1}^n b_k^2
        \geq \left(\sum_{k=1}^n a_k b_k \right)^2
    \end{equation*}
    How to remember: /*TODO*/
\end{tool}
\begin{tool}[Bernoulli's Inequality]
    $(1+y)^n \geq 1 + ny$
\end{tool}
\begin{tool}[Cauchy's Inequality]
    $x\cdot y \leq \frac{1}{2}(x^2+y^2)$
\end{tool}
\begin{tool}[Triangle Inequality]
    $\myabs{x+y} \leq \myabs{x}+\myabs{y}$
\end{tool}
\begin{tool}[Reverse Triangle Inequality]
    $\myabs{x-y} \geq \myabs{\myabs{x}-\myabs{y}}$
\end{tool}
\begin{tool}[Euler's Identity]
    \begin{equation*}
        (x_1^2+y_1^2)(x_2^2+y_2^2) =
        {(x_1x_2-y_1y_2)}^2+{(x_1y_2+x_2y_1)}^2
    \end{equation*}
    How to remember: Interpret as the product of two complex norms.
\end{tool}
\begin{tool}[Euler's Other Identity]
    \begin{align*}
         &(x_1^2+x_2^2+x_3^2+x_4^2)(y_1^2+y_2^2+y_3^2+y_4^2)\\
        =&(x_1y_1-x_2y^2-x_3y_3-x_4y_4)^2+(x_1y_2+x_2y_1+x_3y_4-x_4y_3)^2\\
        +&(x_1y_3-x_2y_4+x_3y_1+x_2y_4)^2+(x_1y_4+x_2y_3-x_3y_2+x_4y_1)^2
    \end{align*}
    How to remember: Interpret as being the product of two quaternion norms.
\end{tool}
\begin{tool}
    \begin{align*}
        a^3+b^3+c^3-3abc
        =\frac{1}{2}(a+b+c)({(a-b)}^2+{(a-c)}^2+{(b-c)}^2)
    \end{align*}
    Can remember this through below determinant. LHS is computation by
    Saurus' rule. RHS is first add columns to first, then compute.
    \begin{equation*}
    \det\begin{pmatrix} a & b & c \\ c & a & b
        \\ b & c & a \end{pmatrix}
    \end{equation*}
\end{tool}
\begin{tool}[Sophie Germain Identity]
    \begin{equation*}
a^4 + 4b^4 = (a^2 + 2b^2 + 2ab)(a^2 + 2b^2 - 2ab)
    \end{equation*}
    /*remember?*/
\end{tool}
\section{Geometry}
\subsection{Approaches}
\subsection{Tools}
\section{Linear Algebra}
\subsection{Approaches}
\subsection{Tools}
\begin{tool}[Cauchy-Schwartz Inequality] /*TODO*/ \end{tool}
\begin{tool}
    $m$ homogeneous linear equations with $n$ variables. $m < n$ implies
    there exists a nontrivial solution.
\end{tool}
\begin{tool}[Trace]
\end{tool}
\begin{tool}[Eigenstuff]
\end{tool}
\begin{tool}
The sum of the eigenvalues is the trace.
\end{tool}
\section{Combinatorics}
\subsection{Approaches}
\begin{approach}
    In proving an identity that takes integer values, find a situation that
    the two halves are counting in different ways.
\end{approach}
\subsubsection{Advanced Combinatorics}
\begin{approach}[Vector Space Dimension Bound Method]
\end{approach}
\begin{approach}[Homogenous System of Linear Equations Method]
\end{approach}
\begin{approach}[Probability Method]
\end{approach}
\subsection{Tools}
\begin{tool}[Pigeonhole Principle]
    If $kn+1$ objects ($k \geq 1$) are distributed among $n$ boxes, one box
    will contain $k+1$ objects. Note that $k$ not necessarily finite, so we
    have an infinite PHP. The notion of equivalence classes is useful for
    grouping things into ``boxes''.
\end{tool}
\begin{tool}[Inclusion-Exclusion Principle] /*TODO*/ \end{tool}
\begin{tool}[Binomial Theorem]
    \begin{equation*}
        (x+y)^n = \sum_{k=0}^{k=n}\begin{pmatrix} n \\ k \end{pmatrix} x^k
        y^{n-k}
    \end{equation*}
\end{tool}
\begin{tool}[Multinomial Theorem] /*TODO and how to remember*/\end{tool}

\section{Analysis, Calculus, and Differential Equations}
\subsection{Approaches}
\begin{approach}
    When trying to prove a claim over $\mathbb{R}$, first prove for
    $\mathbb{N}$, then $\mathbb{Q}$, then $\mathbb{R}$.
\end{approach}
\begin{approach}
    If you have continuity, I think contradiction is solid? /*hash out*/
\end{approach}
\begin{approach}
    Apply definite integrals to differential equations. Example: $g'(x) >
    1$ for all $x$ implies $\int_{t=0}^{t=x}g'(t)dt > \int_{t=0}^{t=x}dt$,
    so $g(x) > x + g(0)$.
\end{approach}
\begin{approach}
    Associate the differential equation $y' = f(y)$ with sequence $a_{n+1}
    - a_n = f(a_n)$. Apparently, this is useful in either direction.
\end{approach}
\subsection{Tools}
\begin{tool}[Fundamental Theorem of Calculus]
    /*2 forms*/\\
    Further, if $f'$ is continuous, then
    \begin{equation*}
        f(x) = f(a) + \int{t=a}^x f'(t) dt
    \end{equation*}
\end{tool}
\begin{tool}[Mean Value Theorem]\end{tool}
\begin{tool}[Intermediate Value Theorem]\end{tool}
\begin{tool}[Taylor's Theorem]\end{tool}
\begin{tool}[Taylor Series]\end{tool}
\begin{tool}[Continuity]\end{tool}
\begin{theorem} Every bounded monotonic sequence is convergent\end{theorem}
\begin{theorem} For every $x \in \mathbb{R}$, there is a rational sequence
$r_n$ with $\lim_{n\to\infty} r_n = x$ \end{theorem}
\begin{tool} Can solve separable differential equations \end{tool}
\begin{tool} Functions that grow at a more than proportional rate to the
function itself blow up in finite time. \end{tool}
\begin{tool}[Additive functions] \end{tool}
\begin{tool}[Hamel Basis]\end{tool}

\section{Set Theory and Abstract Algebra}
\subsection{Approaches}
\begin{approach}
    When possible, impose an order on your set. Why not? It might make it
    easier.
\end{approach}
\subsection{Tools}
\begin{definition}[Order, Total Order]
    An \emph{order} on a set is relation $\leq$ such that
    \begin{enumerate}
        \item $a\leq a$,
        \item $a\leq b$ and $b \leq a$
        \item $(a \leq b) \land (b \leq c) \implies a\leq c$.
    \end{enumerate}
        The order is called \emph{total} if either $a \leq
        b$ or $b \leq a$ for all $a,b$ in the set.
\end{definition}
\section{Problems}
\subsection{Putnam}
\begin{itemize}
\item (B2-2006) Prove that for every set $X = \{x_1, x_2, \dots, x_n\}$ of
    $n$ real numbers, there exists a nonempty subset $S$ of $X$ and an
    integer $m$ such that
    \begin{equation*}
        \big| m+\sum_{s\in S}s \big| \leq \frac{1}{n+1}
    \end{equation*}
    [PHP, fractional part of sum, and chain in $X$]
\end{itemize}
\subsection{Olympiads}
\begin{itemize}
    \item (26th International Mathematical Olympiad, proposed by Mongolia)
        Given a set $M$ of $1985$ distinct positive integers, none of which
        has a prime divisor greater than $26$, prove that $M$ contains at
        least one subset of four distinct elements whose product is the
        fourth power of an integer. [PHP and equivalence classes in
        canonical form].
\end{itemize}
\subsection{Books}
Books drawn upon include: Putnam and Beyond (P\&B), \dots
\begin{itemize}
    \item (P\&B-1) Prove that $\sqrt{2} + \sqrt{3} +\sqrt{5}$ is an
        irrational number. [Contradiction similar to Pythagoras method for
        $\sqrt{2}$] 
    \item (P\&B-2) Show that no set of nine consecutive integers can
        be partitioned into two sets with the product of the elements of
        the first set equal to the product of the elements of the second
        set. [Contradition]
    \item (P\&B-3) Find the least positive integer $n$ such that any set of
        $n$ pairwise relatively prime integers greater than 1 and less than
        2005 contains at least one prime number. [Find an obvious maximal
        example, and argue anything above that exaple is contradiction]
    \item (P\&B-4) Every point of three-dimensional space is colored red,
        green, or blue. Prove that one of the colors attains all distances,
        meaning that any positive real number represents the distance
        between two points of this color. [Contradiction. Careful what the
        negation says, and can solve it without computation]
    \item (P\&B-5) The union of nine planar surfaces, each of area equal to
        1, has a total area equal to 5. Prove that the overlap of some two
        of these surfaces has an area greater than or equal to
        $\frac{1}{9}$.
        [Contradiction, Inclusion-Exclusion approximation]
    \item (P\&B-6) Show that there does not exist a function $f: \mathbb{Z}
        \to \{ 1,2,3\}$ satisfying $f(x) \neq f(y)$ for all $x,y \in
        \mathbb{Z}$ in which $\myabs{x-y} \in \{2,3,5\}$. [Contradition,
        focus on specific numbers].
    \item (P\&B-34) A sequence of $m$ positive integers contains exactly
        $n$ distinct terms. Prove that if $2n \leq m$ then there exists a
        block of consecutive terms whose product is a perfect square. [PHP,
        define relation between products given by even/odd in exponents and
        look at a chain of products in the sequence.]
        
\end{itemize}
\subsection{BSM}
\subsection{School}
\subsection{Other}
\end{document}
